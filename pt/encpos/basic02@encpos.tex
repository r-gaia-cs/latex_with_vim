\chapter{Além do texto puro}
No capítulo anterior introduzimos os comandos mais básicos do LaTeX que
possibilitam o usuário escrever um texto simples. Neste capítulo apresentamos
alguns comandos do LaTeX que são seu diferencial ao escrever textos longos.

\section{Citações}
No LaTeX encontramos dois ambientes dedicados a citações. O primeiro deles é o
\envname{quote}\index{ambiente!quote@\envname{quote}} próprio para citações de
uma única linha e o segundo é o
\envname{quotation}\index{ambiente!quotation@\envname{quotation}} adequado para
citações de vários parágrafos.

\section{Edição direta}
Algumas vezes deseja-se inserir um texto que não deve ser interpretado. Isso é
possível pelo ambiente
\envname{verbatim}\index{ambiente!verbatim@\envname{verbatim}}, coloca o texto
em uma nova linha, e pelo comando
\lstinline!\verb!\index{comando!verb@\lstinline+\verb+}, coloca o texto na mesma
linha.

Tanto o ambiente \lcode{verbatim} como o comando \lstinline!\verb! apresentam
uma fonte própria. \\
\example{codes/hello_verb@latex_with_vim.tex}

Vale destacar que o comando \textbackslash\lcode{verb} é ``flexível'' quando ao
delimitador, os caracteres \lstinline+!+, \lstinline!+! e \lstinline!:!
normalmente exercem satisfatoriamente esta função.

\section{Nota de rodapé}
Para produzir notas de rodapé\index{nota de rodape@nota de rodapé} deve-se
utilizar o comando
\lstinline!\footnote!\index{comando!footnote@\lstinline+\footnote+} que deve
ocorrer imediatamente depois da palavra ou texto a que se refere a nota de
rodapé e como parâmetro do comando o texto a ser inserido na nota de rodapé.

\section{Listas}
Para a construção de listas\index{lista} podemos utilizar um dos quatro
ambientes: \envname{itemize}, \envname{enumerate},
\envname{description}\footnote{Não será tratado neste curso} ou
\envname{list}\footnote{Não será tratado neste curso}. E para a criação de
sublistas basta adicionar um dos ambientes dentro de um já existente.

Cada item de uma lista é identificado, no LaTeX, pelo comando
\lstinline!\item!\index{comando!item@\lstinline+\item+} que deve preceder o
texto.

\subsection{\envname{itemize}}
O ambiente \envname{itemize}\index{ambiente!itemize@\envname{itemize}} utiliza
um símbolo para indicar cada item da lista. \\
\example{codes/itemize@latex_with_vim.tex}

\subsection{\envname{enumerate}}
O ambiente \envname{enumerate}\index{ambiente!enumerate@\envname{enumerate}}
numera cada um dos itens da lista. \\
\example{codes/enumerate@latex_with_vim.tex}

Ao utilizar o ambiente \envname{enumerate} é permitido para cada item adicionar
um comando \lstinline!\label! e posteriormente fazer referência a este pelo
comando \lstinline!\ref!.

\section{Tabelas}
O LaTeX permite construir tabelas\index{tabela} e
adicionar legendas \`{a} estas.

\subsection{\envname{tabular}}
O ambiente \envname{tabular}\index{ambiente!tabular@\envname{tabular}} é
utilizado para a construção de tabelas no LaTeX e sua sintaxe é
\begin{code}
  \begin{tabular}[colunas]
    informacao
  \end{tabular}
\end{code}
onde \lcode{colunas} é uma sequência de caracteres, onde cada caractere
corresponde a uma coluna e o respectivo alinhamento que são apresentados na
Tabela~\ref{tab:par_colunas}, e \lcode{informacao} é o conteúdo de cada célula
da tabela.
\begin{table}[h!tb]
  \centering
  \caption{Opções disponíveis para \envname{colunas}.}
  \label{tab:par_colunas}
  % Filename: tabular_halign@latex_with_vim.tex
% This code is part of LaTeX with Vim.
% 
% Description: LaTeX with Vim is free book about Vim, LaTeX and Git.
% 
% Created: 30.03.12 12:11:31 AM
% Last Change: 30.03.12 12:11:38 AM
% 
% Author: Raniere Gaia Costa da Silva, r.gaia.cs@gmail.com
% Organization:  
% 
% Copyright (c) 2010, 2011, 2012, Raniere Gaia Costa da Silva. All rights 
% reserved.
% 
% This file is license under the terms of a Creative Commons Attribution 
% 3.0 Unported License, or (at your option) any later version. More details
% at <http://creativecommons.org/licenses/by/3.0/>.
\begin{tabular}{lp{0.8\textwidth}}
    \hline
    Código & Descrição \\ \hline
    \envname{l} & Alinha com margem esquerda. \\
    \envname{r} & Alinha com a margem direita. \\
    \envname{c} & Centralizado. \\
    \envname{p} & Requer como parâmetro a largura da columa. \\
    \textbar & Imprime uma linha separando as colunas. \\ \hline
\end{tabular}

\end{table}

Cada célula da tabela deve ser separadas pelo comando
\lstinline!&!\index{comando! @\lstinline+&+} e a mudança de linha ocorre pelo
comando \lstinline!\\!\index{comando! @\lstinline+\\+} ou
\lstinline!\tabularnewline!\index{comando!tabularnewline@\lstinline+\tabularnewline+}.
Para imprimir uma linha horizontal separando duas linhas da tabela deve-se
utilizar o comando \lstinline!\hline!.\\
\example{codes/tabular@latex_with_vim.tex}

Outros comandos também são importantes para a construção mas não trataremos
deles aqui, para conhecê-los visitar
\url{http://en.wikibooks.org/wiki/LaTeX/Tables}.

\subsection{\envname{table}}
O ambiente \envname{table}\index{ambiente!table@\envname{table}} possibilita a
inclusão de uma legenda para a tabela e trabalha a mesma como um objeto
flutuante. A sintaxe deste ambiente é
\begin{code}
  \begin{table}[posicao]
    tabela
    \caption{legenda}
    \label{P:tebela}
  \end{table}
\end{code}
onde \lcode{posicao} é o parâmetro que indica onde a tabela deve ser
preferencialmente inserida (as opções disponíveis são apresentadas na Tabela
\ref{tab:par_place_tab} e a opção padrão é \lcode{tbp}), \lcode{tabela}
corresponde ao código da tabela a ser inserida,
\lstinline!\caption!\index{comando!caption@\lstinline+\caption+} é o comando
correspondente a legenda e \lcode{legenda} é o texto a ser apresentado como
legenda, \lstinline!\label! é o comando para referência cruzada como já
apresentado. \\
\example{codes/table@latex_with_vim.tex}
\begin{table}[!htb]
  \centering
  \caption{Opções disponíveis para \envname{posicao}.}
  \label{tab:par_place_tab}
  % Filename: table_place@latex_with_vim.tex
% This code is part of LaTeX with Vim.
% 
% Description: LaTeX with Vim is free book about Vim, LaTeX and Git.
% 
% Created: 30.03.12 12:11:31 AM
% Last Change: 30.03.12 12:11:38 AM
% 
% Author: Raniere Gaia Costa da Silva, r.gaia.cs@gmail.com
% Organization:  
% 
% Copyright (c) 2010, 2011, 2012, Raniere Gaia Costa da Silva. All rights 
% reserved.
% 
% This file is license under the terms of a Creative Commons Attribution 
% 3.0 Unported License, or (at your option) any later version. More details
% at <http://creativecommons.org/licenses/by/3.0/>.
\begin{tabular}{lp{0.8\textwidth}}
    \hline
    Código & Descrição \\ \hline
    \envname{h} & Na posição onde o código se encontra. \\
    \envname{t} & No topo de uma página. \\
    \envname{b} & No fim de uma página. \\
    \envname{p} & Em uma página separada. \\
    \envname{!} & Modifica algumas configurações a respeito de boa posição para objeto flutuante. \\ \hline
\end{tabular}

\end{table}

Uma dica útil é que o comando
\lstinline!\clearpage!\index{comando!clearpage@\lstinline+\clearpage+} força as
tabelas pendentes a serem inseridas.

\subsection{Extensão Calc2LaTeX}
Muitas vezes temos uma tabela no Calc\footnote{O Calc é um dos aplicativos do
pacote LibreOffice e corresponde ao popular Excel do pacote Microsoft Office.} e
desejamos transportá-la para o LaTeX. Para essa tarefa a extensão/macro
Calc2LaTeX, disponível gratuitamente em
\url{http://extensions.services.openoffice.org/en/project/Calc2LaTeX}, é
bastante eficiente.

\section{Referência cruzada} \label{sse:cross_reference}
Existem dois tipos de referência cruzada\index{referencia cruzada@referência
cruzada}, a primeira para alguma parte do documento e a segunda para um outro
documento. Nesta seção abordaremos o primeiro tipo e o segundo será tratado
quando formos falar sobre o BibTeX.

Para alguns comandos e ambientes o LaTeX atribui um número, ou conjunto de
caracteres, que pode ser vinculado a um nome pelo comando
\lstinline!\label!\index{comando!label@\lstinline+\label+} e referenciado pelo
comando \lstinline!\ref!\index{comando!ref@\lstinline+\ref+} e
\lstinline!\pageref!, este último quando deseja-se o número da página onde
encontra-se o item referenciado.

O argumento do comando \lstinline!\label! é uma sequencia de
caracteres\footnote{Recomenda-se escolher uma sequencia ``amigável''.},
\flang{case sensitive}, que será utilizada como argumento do comando
\lstinline!\ref! ao efetuar a referência.

Ao utilizar os comandos \lstinline!\ref! ou \lstinline!\pageref! é aconselhável
precedê-los por um \lstinline!~! para evitar uma quebra de linha antes da
referência.
