\chapter{Matemática no \LaTeX, amsmath} \label{sch:math}
Neste capítulo abordaremos o modo matemático do LaTeX, com uma ênfase nos
pacotes amsmath\index{pacote!amsmath@\pkgname{amsmath}}, amsfonts, amssymb e
amsthm.

\section{Modo matemático}
Para que expressões matemáticas seja processadas corretamente, deve-se mudar do
modo texto para o modo matemático, o que pode ser feito de várias maneiras.

A apresentação de expressões matemáticas pode ocorrer de duas maneiras:
\flang{inline}\index{modo matematico@modo matemático!inline@\flang{inline}},
quando aparecem na mesma linha do texto, e \flang{displayed}\index{modo
matematico@modo matemático!displayed@\flang{displayed}}, quando aparecem em uma
linha própria e centralizada (podendo ou não ser numerada\footnote{Deve-se
numerar apenas equações as quais serão feita referências posteriormente.}).

A seguir, informaremos como proceder para produzir expressões matemáticas
\flang{inline} ou \flang{displayed}. Ao final, apresentaremos algumas dicas
sobre o uso de expressões \flang{inline} e \flang{displayed}.

\subsection{\flang{Inline}}
Expressões matemáticas \flang{inline}\index{modo matematico@modo
matemático!inline@\flang{inline}} devem ser iniciadas por \lstinline!$! e
fechadas por \lstinline!$! ou iniciadas por \lstinline!\)! e fechadas por
\lstinline!\)!. \\
\example{codes/math_inline@latex_with_vim.tex}

\subsection{\flang{Displayed}}
Expressões matemáticas \flang{displayed}\index{modo matematico@modo
matemático!displayed@\flang{displayed}} devem ser iniciadas por \lstinline!$$! e
fechadas por \lstinline!$$! ou iniciadas por \lstinline!\[! e fechadas por
\lstinline!\]!. \\
\example{codes/math_display@latex_with_vim.tex}

Alguns ambientes, como \envname{equation}, \envname{eqnarray} e \envname{align},
também produzem expressões matemáticas \flang{displayed}.

\subsection{Uso de \flang{inline} e \flang{displayed}}
Um ótimo resumo sobre quando usar expressões \flang{inline} e \flang{displayed}
encontra-se em
\url{http://www.math.uiuc.edu/~hildebr/tex/displays.html}\nocite{Hildebrand:TeX_Resoures}
e a seguir apresentaremos tradução de alguns trechos. Para maiores detalhes
recomenda-se uma leitura na obra ``Mathematics Into
Type''\nocite{Swanson:1999:Mathematics}.

Expressões \flang{inline} são ``feias'' quando apresentam frações, somatórios,
integrais, \ldots e algumas vezes precisam de um cuidado especial para
respeitarem as margens. Entretanto, deve-se preferir utilizar expressões
\flang{displayed} apenas nas seguintes ocasiões:
\begin{itemize}
  \item a expressão é longa (ocupa mais da metade de uma linha);
  \item a expressão requer bastante espaço vertical, i.e., possui várias
    frações, somatórios, integrais, \ldots;
  \item a equação será numerada;
  \item a expressão que você deseja destacar/enfatizar.
\end{itemize}

\section{Primeiros comandos no modo matemático}
A seguir enunciaremos como proceder para produzir as primeiras equações, mas
antes é importante saber que o modo matemático ignora qualquer espaço (para
inserir um espaço em branco no modo matemático veja a seção
\ref{sss:math:textos_e_espacamentos}).

\subsection{Operações aritméticas básicas}
As operações aritméticas básicas\index{modo matematico@modo matemático!operacoes
aritmeticas basicas@operações aritméticas básicas} são escritas normalmente,
exceto pela multiplicação que utiliza-se dos comandos \lstinline!\times! ou
\lstinline!\cdot!\footnote{O uso do comando mais adequado depende muito do campo
de estudo.} e das frações representada pelo comando
\lstinline!\frac!\footnote{Deve-se ponderar o uso deste comando por questão de
legibilidade.}. \\
\example{codes/math_powers_indices@latex_with_vim.tex}

\subsection{Índices e expoentes}
Índices\index{modo matematico@modo matemático!indice@índice} e
expoentes\index{modo matematico@modo matemático!expoente} são indicados pelos
respectivos comandos: \flang{underscore}, \lcode{_}, e \flang{caret}, \lcode{^}.
Por padrão apenas o primeiro símbolo depois do comando é alterado, quando for
necessário mais de um símbolo deve-se utilizar chaves.

O símbolo \flang{prime}, muito utilizado para derivadas, já vem posicionado
corretamente.\footnote{Algumas vezes deve-se preferir utilizar o comando
\lcode{prime} em conjunto com \flang{underscore} e/ou \flang{caret}.} \\
\example{codes/math_powers_indices@latex_with_vim.tex}

\subsection{Acentos}
Os acentos\index{modo matematico@modo matemático!acento} disponíveis no modo
matemático são apresentados na Tabela~\ref{tab:math_accents}.
\begin{table}[h!tb]
  \centering
  \caption{Acentos disponíveis no modo matemático.}
  \label{tab:math_accents}
  % Filename: math_accents@latex_with_vim.tex
% This code is part of LaTeX with Vim.
% 
% Description: LaTeX with Vim is free book about Vim, LaTeX and Git.
% 
% Created: 30.03.12 12:14:55 AM
% Last Change: 30.03.12 12:15:01 AM
% 
% Author: Raniere Gaia Costa da Silva, r.gaia.cs@gmail.com
% Organization:  
% 
% Copyright (c) 2010, 2011, 2012, Raniere Gaia Costa da Silva. All rights 
% reserved.
% 
% This file is license under the terms of a Creative Commons Attribution 
% 3.0 Unported License, or (at your option) any later version. More details
% at <http://creativecommons.org/licenses/by/3.0/>.
\begin{tabular}{cc|cc|cc}
    \hline
    Comando & Resultado & Comando & Resultado & Comando & Resultado \\ \hline
    \textbackslash\textsf{acute\{a\}} & $\acute{a}$ & \textbackslash\textsf{bar\{a\}} & $\bar{a}$ & \textbackslash\textsf{breve\{a\}} & $\breve{a}$ \\
    \textbackslash\textsf{check\{a\}} & $\check{a}$ & \textbackslash\textsf{dot\{a\}} & $\dot{a}$ & \textbackslash\textsf{ddot\{a\}} & $\ddot{a}$ \\
    \textbackslash\textsf{dddot\{a\}} & $\dddot{a}$ & \textbackslash\textsf{ddddot\{a\}} & $\ddddot{a}$ & \textbackslash\textsf{grave\{a\}} & $\grave{a}$ \\
    \textbackslash\textsf{hat\{a\}} & $\hat{a}$ & \textbackslash\textsf{widehat\{a\}} & $\widehat{a}$ & \textbackslash\textsf{mathring\{a\}} & $\mathring{a}$ \\
    \textbackslash\textsf{tilde\{a\}} & $\tilde{a}$ & \textbackslash\textsf{widetilde\{a\}} & $\widetilde{a}$ & \textbackslash\textsf{vec\{a\}} & $\vec{a}$ \hfill
\end{tabular}

\end{table}

\subsection{Delimitadores}
Parênteses\index{modo matematico@modo
matemático!parenteses@parênteses|see{delimitadores}}, colchetes\index{modo
matematico@modo matemático!colchetes|see{delimitadores}} e chaves\index{modo
matematico@modo matemático!chaves|see{delimitadores}} são exemplos de
delimitadores\index{modo matematico@modo matemático!delimitadores}. Uma lista
completa dos delimitadores disponíveis no LaTeX encontra-se na
Tabela~\ref{tab:math_delimiter}.
\begin{table}[h!tb]
  \centering
  \caption{Delimitadores disponíveis no LaTeX.}
  \label{tab:math_delimiter}
  % Filename: math_delimiter@latex_with_vim.tex
% This code is part of LaTeX with Vim.
% 
% Description: LaTeX with Vim is free book about Vim, LaTeX and Git.
% 
% Created: 30.03.12 12:16:24 AM
% Last Change: 30.03.12 12:16:29 AM
% 
% Author: Raniere Gaia Costa da Silva, r.gaia.cs@gmail.com
% Organization:  
% 
% Copyright (c) 2010, 2011, 2012, Raniere Gaia Costa da Silva. All rights 
% reserved.
% 
% This file is license under the terms of a Creative Commons Attribution 
% 3.0 Unported License, or (at your option) any later version. More details
% at <http://creativecommons.org/licenses/by/3.0/>.
\begin{tabular}{>{\centering}p{0.13\linewidth}<{\centering}>{\centering}p{0.07\linewidth}<{\centering}|>{\centering}p{0.13\linewidth}<{\centering}>{\centering}p{0.07\linewidth}<{\centering}|>{\centering}p{0.13\linewidth}<{\centering}>{\centering}p{0.07\linewidth}<{\centering}|>{\centering}p{0.13\linewidth}<{\centering}>{\centering}p{0.07\linewidth}<{\centering}}
    \hline
    Com. & Res. & Com. & Res. & Com. & Res. & Com. & Res. \tabularnewline \hline
    \lstinline!(! & $($ & \lstinline!)! & $)$ & \lstinline![! & $[$ & \lstinline!]! & $]$ \tabularnewline
    \lstinline!\{! & $\{$ & \lstinline!\}! & $\}$ & \lstinline!\backslash! & $\backslash$ & \lstinline!/! & $/$ \tabularnewline
    \lstinline!\langle! & $\langle$ & \lstinline!\rangle! & $\rangle$ & \lstinline!|! & $|$ & \lstinline!\|! & $\|$ \tabularnewline
    \lstinline!\lfloor! & $\lfloor$ & \lstinline!\rfloor! & $\rfloor$ & \lstinline!\lceil! & $\lceil$ & \lstinline!\rceil! & $\rceil$ \tabularnewline
    \lstinline!\ulcorner! & $\ulcorner$ & \lstinline!\urcorner! & $\urcorner$ &\lstinline!\llcorner! & $\llcorner$ & \lstinline!\lrcorner! & $\lrcorner$ \tabularnewline \hline
\end{tabular}
\begin{flushleft}
    \textbf{Nota:} Enquanto que \lstinline!|! \'{e} um limitador \lstinline!\mid! \'{e} um operador l\'{o}gico.
\end{flushleft}

\end{table}

Para expressões matemáticas no modo \flang{displayed} ou longas é aconselhável
utilizar os comandos \lstinline!\left! e \lstinline!\right! anteriormente ao
limitador para ajustá-lo verticalmente. \\
\example{codes/math_delimiters_sizes@latex_with_vim.tex}

\subsection{Textos e espaçamentos} \label{sss:math:textos_e_espacamentos}
Existem três ocasiões em que é preciso inserir um texto\index{modo
matematico@modo matemático!texto} dentro de uma expressão matemática:
\begin{itemize}
  \item um operador matemático é representado pelas primeiras letras de seu nome, e.g., $\max$, $\min$, $\lim$, \ldots;
  \item uma variável é representada por mais de uma letra;
  \item incluir uma explicação/justificativa.
\end{itemize}

O LaTeX já possui vários operadores matemáticos definidos (são apresentados mais
a frente) e quando o operador desejado não estiver definido deve-se utilizar o
comando \lstinline!\operatorname!\index{modo matematico@modo matemático!novos
operadores} ou \lstinline!\DeclareMathOperator!, este último quando o operador
for ser utilizado várias vezes no documento. 

Em relação ao nome de variáveis\index{modo matematico@modo matemático!nomes
longos para variaveis@nomes longos para variáveis}, deve-se evitar ao máximo
nomeá-las com mais de uma letra (utilizar o alfabeto grego para isso). Quando
não for possível evitar, deve-se utilizar o comando \lstinline!\mathrm! para
evitar confusões. \\
\example{codes/math_var_names@latex_with_vim.tex}

Já para a inclusão de textos explicativos deve-se utilizar o comando
\lstinline!\text!\index{comando!text@\lstinline+\text+} e
\lstinline!\intertext!, este último reservado apenas para expressões
\flang{displayed}. \\
\example{codes/math_text@latex_with_vim.tex}

Quanto ao espaçamento\index{modo matematico@modo
matemático!espacamento@espaçamento}, normalmente não é preciso se preocupar com
este pois o LaTeX inclui o espaçamento adequado. Em raras ocasiões deve-se
incluir algum espaço apresentado na Tabela~\ref{tab:math_spacing}.
\begin{table}[!htb]
  \centering
  \caption{Espaçamento no modo matemático.}
  % Filename: algorithmic@latex_with_vim.tex
% This code is part of LaTeX with Vim.
% 
% Description: LaTeX with Vim is free book about Vim, LaTeX and Git.
% 
% Created: 30.03.12 12:11:31 AM
% Last Change: 30.03.12 12:11:38 AM
% 
% Author: Raniere Gaia Costa da Silva, r.gaia.cs@gmail.com
% Organization:  
% 
% Copyright (c) 2010, 2011, 2012, Raniere Gaia Costa da Silva. All rights 
% reserved.
% 
% This file is license under the terms of a Creative Commons Attribution 
% 3.0 Unported License, or (at your option) any later version. More details
% at <http://creativecommons.org/licenses/by/3.0/>.
\begin{tabular}{>{\centering}p{0.1\linewidth}<{\centering}>{\centering}p{0.15\linewidth}<{\centering}>{\centering}p{0.15\linewidth}<{\centering}|>{\centering}p{0.1\linewidth}<{\centering}>{\centering}p{0.15\linewidth}<{\centering}>{\centering}p{0.15\linewidth}<{\centering}}
    \hline
    Abrev. & Comando & Exemplo & Abrev. & Comando & Exemplo \tabularnewline \hline
    & sem espa\c{c}o & $\Rightarrow \Leftarrow$ & \lstinline!\,! & \lstinline!\thinspace! & $\Rightarrow \, \Leftarrow$ \tabularnewline
    \lstinline!\:! & \lstinline!\medspace! & $\Rightarrow \; \Leftarrow$ & \lstinline!\;! & \lstinline!\thickspace! & $\Rightarrow \; \Leftarrow$ \tabularnewline
    & \lstinline!\quad! & $\Rightarrow \quad \Leftarrow$ & & \lstinline!\qquad! & $\Rightarrow \qquad \Leftarrow$ \tabularnewline \hline
\end{tabular}

  \label{tab:math_spacing}
\end{table}
Uma dessas ocasiões é em integrais. \\
\example{codes/math_space_integral@latex_with_vim.tex}

\subsection{Matrizes}
Para a construção de matrizes\index{modo matematico@modo matemático!matrizes} (e
vetores\index{modo matematico@modo matemático!vetores|see{matrizes}}) utiliza-se
o ambiente \envname{matrix} onde as colunas são separadas por \lstinline!&! e as
linhas por \lstinline!\\!. \\
\example{codes/math_matrix@latex_with_vim.tex}

Destaca-se que o ambiente \envname{matrix} só pode ser utilizado dentro do
ambiente matemático e que na última linha não utiliza-se o comando
\lstinline!\\!\index{comando! @\lstinline+\\+}.

Pode-se utilizar limitadores envolvendo o ambiente \envname{matrix} ou utilizar
uma variante: \envname{pmatrix}, \envname{bmatrix}, \envname{Bmatrix},
\envname{vmatrix} ou \envname{Vmatrix} que corresponde, respectivamente, aos
delimitadores $( )$, $[ ]$, $\{ \}$, $| |$ e $\| \|$. \\

\section{Comandos avançados no modo matemático}
\subsection{Equações, numeração e referenciação} \label{sse:latex:equation}
Para o uso de expressões matemáticas a serem referenciadas\index{modo
matematico@modo matemático!numeracao@numeração} posteriormente, recomenda-se o
ambiente \envname{equation}\index{ambiente!equation@\envname{equation}} em
conjunto com o comando
\lstinline!\label!\index{comando!label@\lstinline+\label+}. \\
\example{codes/math_equation@latex_with_vim.tex}

No exemplo acima, \lcode{E:TeoPit} correspondente ao parâmetro do comando
\lstinline!\label!, como apresentado na Seção~\ref{sse:cross_reference}. A
referência a equação ocorre pelo comando \lstinline!\eqref!. \\
\example{codes/math_equation_eqref@latex_with_vim.tex}

\subsection{\flang{Tags}}
O comando \lstinline!\tag!\index{comando!tag@\lstinline+\tag+} do LaTeX nomeia
uma equação\index{modo matematico@modo matemático!tag@\flang{tag}} e a
referência passa a ser feito por este. \\
\example{codes/math_equation_tag@latex_with_vim.tex}

Vale destacar que podemos utilizar o comando \lstinline!\label! como parâmetro
do comando \lstinline!\tag!.

\subsection{Teorema}
O comando \lstinline{\newtheorem}\index{modo matematico@modo matemático!teorema}
deve ser inserido no \textit{preâmbulo} e é responsável por criar um ambiente
numerado para informações. Sua sintaxe é
\begin{code}
  \newtheorem{nome}{texto}
\end{code}
onde \lcode{nome} é o nome do ambiente a ser criado e \lcode{texto} é a
sequência de caracteres que precede a numeração. Caso deseje-se não numerar
deve-se utilizar a sintaxe
\begin{code}
  \newtheorem*{nome}{texto}
\end{code}

Para fazer uso do novo ambiente deve-se utilizar a sintaxe padrão para um
ambiente
\begin{code}
  \begin{nome}
    ...
  \end{nome}
\end{code}
ou ainda
\begin{code}
  \begin{nome}[XXX]
    ...
  \end{nome}
\end{code}
onde \lcode{XXX} é uma sequência de caracteres que aparece entre parênteses logo
após a numeração.

\subsection{Demonstração}
O ambiente \envname{proof} é destinada a demonstrações\index{modo
matematico@modo matemático!demonstracao@demonstração} e caracterizado por
terminar com o comando \lstinline!\qed!. \\
\example{codes/math_proof@latex_with_vim.tex}

O ambiente \envname{proof}, como podemos observar no exemplo abaixo, não
trabalha adequadamente quando é finalizado com uma expressão matemática
\flang{displayed} e para corrigir isso devemos informar onde onde será inserido
o símbolo \flang{qed}. \\
\example{codes/math_proof_set_qed@latex_with_vim.tex}

\subsection{Alinhamento}
O ambiente \envname{equation} foi projetado para trabalhar apenas com equações
de uma única linha, nesta seção vamos apresentar algumas formas de trabalhar com
equações com várias linhas\index{modo matematico@modo matemático!multiplas
equacoes@múltiplas equações}.

Para múltiplas equações alinhadas utilizamos o ambiente
\envname{align}\index{ambiente!align@\envname{align}}, sendo cada linha separada
pelo comando \lstinline!\\!\index{comando! @\lstinline+\\+} e o alinhamento por
\lstinline!&!\index{comando! @\lstinline+&+}. \\
\example{codes/math_align@latex_with_vim.tex}

Quando o alinhamento ocorrer adjacente a um sinal de \lstinline!=!,
\lstinline!+!, \dots devemos utilizar o comando \lstinline!&! antes do sinal.

O ambiente \envname{align} numera todas as equações. Caso não queira numerar uma
ou mais equações deve-se utilizar o comando \lstinline!\notag! em cada linha
correspondente.

O comando \lstinline!\label! deve estar presente em cada linha.

Quando desejar adicionar a alguma linha alguma anotação utiliza-se o comando
\lstinline!&&! entre a equação e a anotação. \\
\example{codes/math_align_annotation@latex_with_vim.tex}

\subsection{Fórmulas longas}
Fórmulas muito longas é fonte de vários problemas ao utilizar o LaTeX. Se
existir fórmulas muito longas na obra que estiver trabalhando sugere-se inserir
o pacote \pkgname{breqn} por este quebrá-las automaticamente ao utilizar o
ambiente \envname{dmath} no lugar de \envname{equation}.

Infelizmente o pacote \pkgname{breqn} nem sempre funciona como desejado e nesses
casos a solução é fazer a quebra da equação manualmente. Para isso, deve-se
utilizar o ambiente \envname{multline}, para uma única equação, ou
\envname{split}, este último deve ser utilizado dentro de um outro ambiente
matemático. Se for quebrar as equações manualmente, recomenda-se ler a seção
``Split equations without alignment'' de ``User’s Guide for the amsmath Package''.

\subsection{Ocultando termos}
Ao trabalhar com fórmulas muito longas tenta-se diminuir o tamanho utilizando
sequências e muitas vezes é aconselhável indicar o número de termos. Para isso
podemos utilizar os comandos \lstinline!\overbrace! ou \lstinline!\underbrace!.
\\
\example{codes/math_underbrace@latex_with_vim.tex}

\subsection{Funções definidas por partes}
É relativamente comum definirmos uma equações por partes e o ambiente adequado
para representar esta construção é o \envname{cases}\index{modo matematico@modo
matemático!funcoes definidas por partes@funções definidas por partes}. \\
\example{codes/math_cases@latex_with_vim.tex}

O ambiente \envname{cases}\index{modo matematico@modo matemático!sistemas de
equacoes@sistemas de equações} também pode ser utilizado para sistemas de
equações.

\subsection{Fonte e Símbolos}
No modo matemático, o LaTeX classifica os caracteres em alfabeto matemático e
símbolos matemáticos. Baseado nessa classificação escolhe uma fonte a ser usada.

Para alterar a fonte de caracteres do alfabeto matemático utiliza-se o comando
\lstinline!\mathXX! sendo que \lcode{XX} corresponde ao código da fonte a ser
utilizada. A Tabela~\ref{tab:op_amfonte} apresenta alguns das opções
disponíveis.
\begin{table}[h!tb]
  \centering
  \caption{Opções disponíveis para \lcode{XX} da fonte para o alfabeto matemático.}
  \label{tab:op_amfonte}
  % Filename: math_op_amfonte@latex_with_vim.tex
% This code is part of LaTeX with Vim.
% 
% Description: LaTeX with Vim is free book about Vim, LaTeX and Git.
% 
% Created: 30.03.12 12:11:31 AM
% Last Change: 30.03.12 12:11:38 AM
% 
% Author: Raniere Gaia Costa da Silva, r.gaia.cs@gmail.com
% Organization:  
% 
% Copyright (c) 2010, 2011, 2012, Raniere Gaia Costa da Silva. All rights 
% reserved.
% 
% This file is license under the terms of a Creative Commons Attribution 
% 3.0 Unported License, or (at your option) any later version. More details
% at <http://creativecommons.org/licenses/by/3.0/>.
\begin{tabular}{lp{0.8\textwidth}}
    \hline
    Código & Descrição \\ \hline
    \lcode{it} & Texto em itálico. \\
    \lcode{bf} & Texto em negrito. \\
    \lcode{rm} & Texto em romano. \\
    \lcode{sf} & Texto em sans serif. \\
    \lcode{tt} & Texto na tipografia de uma máquina de escrever.
\end{tabular}

\end{table}

A seguir é ilustrado as opções apresentadas na Tabela \ref{tab:op_amfonte}. \\
\example{codes/math_op_amfonte@latex_with_vim.tex}

Para símbolos matemáticos apenas é possível apresentá-los em negrito e, para
isso, utiliza-se o comando \lstinline!\boldsymbol!. \\
\example{codes/math_boldsymbol@latex_with_vim.tex}

No LaTeX também existe quatro alfabetos que são interpretados como símbolos. Um
deles é o alfabeto grego, apresentado no capítulo anterior e os outros três são
acessados com o comando \lstinline!\mathXX!, sendo que \lcode{XX} corresponde ao
código da fonte a ser utilizada. A Tabela~\ref{tab:op_asfonte} apresenta as
opções disponíveis.
\begin{table}[h!tb]
  \centering
  \caption{Opções disponíveis para \lcode{XX} da fonte para o alfabeto matemático interpretado como símbolo.}
  \label{tab:op_asfonte}
  % Filename: math_op_asfonte@latex_with_vim.tex
% This code is part of 'LaTeX with Vim'.
% 
% Description: This file correspond to a example of LaTeX producing newlines.
% 
% Created: 27.06.12 08:29:06 AM
% Last Change: 27.06.12 08:29:06 AM
% 
% Author: Raniere Gaia Costa da Silva, r.gaia.cs@gmail.com
% 
% Copyright (c) 2012, Raniere Gaia Costa da Silva. All rights 
% reserved.
% 
% This work is licensed under the Creative Commons Attribution-ShareAlike 3.0 Unported License. To view a copy of this license, visit http://creativecommons.org/licenses/by-sa/3.0/ or send a letter to Creative Commons, 444 Castro Street, Suite 900, Mountain View, California, 94041, USA.
%
\begin{tabular}{lp{0.8\textwidth}}
    \hline
    Código & Descrição \\ \hline
    \lcode{cal} & Texto em caligráfico, apenas para caixa alta. \\
    \lcode{frak} & Texto em Euler Fraktur. \\
    \lcode{bb} & Texto em blackboard bold, apenas para caixa alta.
\end{tabular}

\end{table}

A seguir é ilustrado as opções apresentadas na Tabela~\ref{tab:op_asfonte}. \\
\example{codes/math_op_asfonte@latex_with_vim.tex}

Destaca-se que a fonte blackboard bold é normalmente utilizada para representar
os conjuntos dos números naturais ($\mathbb{N}$), inteiros ($\mathbb{Z}$), reais
($\mathbb{R}$) e complexos ($\mathbb{C}$).

\section{Símbolos e operadores}
A seguir apresentaremos vários dos símbolos e operadores disponíveis no LaTeX.
Para uma lista completa recomenda-se ``The Comprehensive LaTeX Symbol
List''\nocite{Pakin:2009:Symbol}. Ao final, abordamos os comandos para raiz
quadrada, binomial e congruências.
\begin{table}[h!tb]
  \centering
  \caption{Setas}
  \label{tab:math_arrows}
  % Filename: math_arrows@latex_with_vim.tex
% This code is part of LaTeX with Vim.
% 
% Description: LaTeX with Vim is free book about Vim, LaTeX and Git.
% 
% Created: 30.03.12 12:15:12 AM
% Last Change: 30.03.12 12:15:22 AM
% 
% Author: Raniere Gaia Costa da Silva, r.gaia.cs@gmail.com
% Organization:  
% 
% Copyright (c) 2010, 2011, 2012, Raniere Gaia Costa da Silva. All rights 
% reserved.
% 
% This file is license under the terms of a Creative Commons Attribution 
% 3.0 Unported License, or (at your option) any later version. More details
% at <http://creativecommons.org/licenses/by/3.0/>.
\begin{tabular}{cc|cc|cc}
    \hline
    Com. & Res. & Com. & Res. & Com. & Res. \\ \hline
    \lstinline!\leftarrow! & $\leftarrow$ & \lstinline!\rightarrow! & $\rightarrow$ & \lstinline!\longleftarrow! & $\longleftarrow$ \\
    \lstinline!\longrightarrow! & $\longrightarrow$ & \lstinline!\Leftarrow! & $\Leftarrow$ & \lstinline!\Rightarrow! & $\Rightarrow$ \\
    \lstinline!\Longleftarrow! & $\Longleftarrow$ & \lstinline!\Longrightarrow! & $\Longrightarrow$ & \lstinline!\nleftarrow! & $\nleftarrow$ \\
    \lstinline!\nrightarrow! & $\nrightarrow$ & \lstinline!\nLeftarrow! & $\nLeftarrow$ & \lstinline!\nRightarrow! & $\nRightarrow$ \\
    \lstinline!\leftrightarrow! & $\leftrightarrow$ & \lstinline!\longleftrightarrow! & $\longleftrightarrow$ & \lstinline!\Leftrightarrow! & $\Leftrightarrow$ \\
    \lstinline!\Longleftrightarrow! & $\Longleftrightarrow$ & \lstinline!\nleftrightarrow! & $\nleftrightarrow$ & \lstinline!\nLeftrightarrow! & $\nLeftrightarrow$ \\
    \lstinline!\dashleftarrow! & $\dashleftarrow$ & \lstinline!\dashrightarrow! & $\dashrightarrow$ & \lstinline!\leftrightharpoons! & $\leftrightharpoons$ \\
    \lstinline!\rightleftharpoons! & $\rightleftharpoons$ & \lstinline!\leftrightarrows! & $\leftrightarrows$ & \lstinline!\rightleftarrows! & $\rightleftarrows$ \\
    \lstinline!\mapsto! & $\mapsto$ & \lstinline!\longmapsto! & $\longmapsto$ & \lstinline!\iff! & $\iff$ \\
    \lstinline!\uparrow! & $\uparrow$ & \lstinline!\downarrow! & $\downarrow$ & \lstinline!\Uparrow! & $\Uparrow$ \\
    \lstinline!\Downarrow! & $\Downarrow$ & \lstinline!\updownarrow! & $\updownarrow$ & \lstinline!\Updownarrow! & $\Updownarrow$ \\
    \lstinline!\Lsh! & $\Lsh$ & \lstinline!\Rsh! & $\Rsh$ & \lstinline!\curvearrowleft! & $\curvearrowleft$ \\
    \lstinline!\curvearrowright! & $\curvearrowright$ & \lstinline!\circlearrowleft! & $\circlearrowleft$ & \lstinline!\circlearrowright! & $\circlearrowright$ \\ \hline
\end{tabular}

\end{table}
\begin{table}[!htbp]
  \caption{Relações binárias} \centering
  \label{tab:math_binary_relations}
  % Filename: math_binary_relations@latex_with_vim.tex
% This code is part of LaTeX with Vim.
% 
% Description: LaTeX with Vim is free book about Vim, LaTeX and Git.
% 
% Created: 30.03.12 12:15:58 AM
% Last Change: 30.03.12 12:16:09 AM
% 
% Author: Raniere Gaia Costa da Silva, r.gaia.cs@gmail.com
% Organization:  
% 
% Copyright (c) 2010, 2011, 2012, Raniere Gaia Costa da Silva. All rights 
% reserved.
% 
% This file is license under the terms of a Creative Commons Attribution 
% 3.0 Unported License, or (at your option) any later version. More details
% at <http://creativecommons.org/licenses/by/3.0/>.
\begin{tabular}{cc|cc|cc}
    \hline
    Com. & Res. & Com. & Res. & Com. & Res. \\ \hline
    \lstinline!<! & $<$ & \lstinline!\nless! & $\nless$ & \lstinline!>! & $>$ \\
    \lstinline!\ngtr! & $\ngtr$ & \lstinline!\ll! & $\ll$ & \lstinline!\lll! & $\lll$ \\
    \lstinline!\gg! & $\gg$ & \lstinline!\ggg! & $\ggg$ & \lstinline!=! & $=$ \\
    \lstinline!\neq! & $\neq$ & \lstinline!:! & $:$ & \lstinline!\doteq! & $\doteq$ \\
    \lstinline!\sim! & $\sim$ & \lstinline!\nsim! & $\nsim$ & \lstinline!\cong! & $\cong$ \\
    \lstinline!\ncong! & $\ncong$ & \lstinline!\simeq! & $\simeq$ & \lstinline!\approx! & $\approx$ \\
    \lstinline!\equiv! & $\equiv$ & \lstinline!\leq! ou \lstinline!\le! & $\leq$ & \lstinline!\nleq! & $\nleq$ \\
    \lstinline!\geq! ou \lstinline!\ge! & $\geq$ & \lstinline!\ngeq! & $\ngeq$ & \lstinline!\leqslant! & $\leqslant$ \\
    \lstinline!\nleqslant! & $\nleqslant$ & \lstinline!\geqslant! & $\geqslant$ & \lstinline!\ngeqslant! & $\ngeqslant$ \\
    \lstinline!\eqslantless! & $\eqslantless$ & \lstinline!\eqslantgtr! & $\eqslantgtr$ & \lstinline!\leqq! & $\leqq$ \\
    \lstinline!\nleqq! & $\nleqq$ & \lstinline!\geqq! & $\geqq$ & \lstinline!\ngeqq! & $\ngeqq$ \\
    \lstinline!\lesssim! & $\lesssim$ & \lstinline!\lessapprox! & $\lessapprox$ & \lstinline!\gtrsim! & $\gtrsim$ \\
    \lstinline!\gtrapprox! & $\gtrapprox$ & \lstinline!\prec! & $\prec$ & \lstinline!\nprec! & $\nprec$ \\
    \lstinline!\succ! & $\succ$ & \lstinline!\nsucc! & $\nsucc$ & \lstinline!\preceq! & $\preceq$ \\
    \lstinline!\npreceq! & $\npreceq$  & \lstinline!\succeq! & $\succeq$ & \lstinline!\nsucceq! & $\nsucceq$ \\
    \lstinline!\in! & $\in$ & \lstinline!\notin! & $\notin$ & \lstinline!\owns! & $\owns$ \\
    \lstinline!\subset! & $\subset$ & \lstinline!\supset! & $\supset$ & \lstinline!\subseteq! & $\subseteq$ \\
    \lstinline!\nsubseteq! & $\nsubseteq$ & \lstinline!\supseteq! & $\supseteq$ & \lstinline!\nsupseteq! & $\nsupseteq$ \\
    \lstinline!\subseteqq! & $\subseteqq$ & \lstinline!\nsubseteqq! & $\nsubseteqq$ & \lstinline!\supseteqq! & $\supseteqq$ \\
    \lstinline!\nsupseteqq! & $\nsupseteqq$ & \lstinline!\sqsubset! & $\sqsubset$ & \lstinline!\sqsubseteq! & $\sqsubseteq$ \\
    \lstinline!\sqsupset! & $\sqsupset$ & \lstinline!\sqsupseteq! & $\sqsupseteq$ & \lstinline!\smile! & $\smile$ \\
    \lstinline!\smallsmile! & $\smallsmile$ & \lstinline!\frown! & $\frown$ & \lstinline!\smallfrown! & $\smallfrown$ \\
    \lstinline!\perp! & $\perp$ & \lstinline!\models! & $\models$ & \lstinline!\mid! & $\mid$ \\
    \lstinline!\nmid! & $\nmid$ & \lstinline!\parallel! & $\parallel$ & \lstinline!\nparallel! & $\nparallel$ \\
    \lstinline!\shortmid! & $\shortmid$ & \lstinline!\nshortmid! & $\nshortmid$ & \lstinline!\shortparallel! & $\shortparallel$ \\
    \lstinline!\nshortparallel! & $\nshortparallel$ & \lstinline!\vdash! & $\vdash$ & \lstinline!\nvdash! & $\nvdash$ \\
    \lstinline!\dashv! & $\dashv$ & \lstinline!\vDash! & $\vDash$ & \lstinline!\nvDash! & $\nvDash$ \\
    \lstinline!\Vdash! & $\Vdash $ & \lstinline!\nVdash! & $\nVdash$ & \lstinline!\propto! & $\propto$ \\
    \lstinline!\asymp! & $\asymp$ & \lstinline!\bowtie! & $\bowtie$ & \lstinline!\Join! & $\Join$ \\
    \lstinline!\vartriangleleft! & $\vartriangleleft$ & \lstinline!\ntriangleleft! & $\ntriangleleft$ & \lstinline!\vartriangleright! & $\vartriangleright$ \\
    \lstinline!\ntriangleright! & $\ntriangleright$ & \lstinline!\trianglelefteq! & $\trianglelefteq$ & \lstinline!\ntrianglelefteq! & $\ntrianglelefteq$ \\
    \lstinline!\trianglerighteq! & $\trianglerighteq$ & \lstinline!\ntrianglerighteq! & $\ntrianglerighteq$ &  \lstinline!\blacktriangleleft! & $\blacktriangleleft$ \\
    \lstinline!\blacktriangleright! & $\blacktriangleright$ & \lstinline!\between! & $\between$ & \lstinline!\pitchfork! & $\pitchfork$ \\
    \lstinline!\therefore! & $\therefore$ & \lstinline!\because! & $\because$ \\ \hline
\end{tabular}
\begin{flushleft}
    Enquanto que \lstinline!|! \'{e} um limitador, \lstinline!\mid! \'{e} um operador que corresponde a express\~{a}o ``tal que''.
\end{flushleft}

\end{table}
\begin{table}[!htb]
  \centering
  \caption{Operadores binários}
  \label{tab:math_binary_operations}
  % Filename: math_binary_operations@latex_with_vim.tex
% This code is part of LaTeX with Vim.
% 
% Description: LaTeX with Vim is free book about Vim, LaTeX and Git.
% 
% Created: 30.03.12 12:15:37 AM
% Last Change: 30.03.12 12:15:42 AM
% 
% Author: Raniere Gaia Costa da Silva, r.gaia.cs@gmail.com
% Organization:  
% 
% Copyright (c) 2010, 2011, 2012, Raniere Gaia Costa da Silva. All rights 
% reserved.
% 
% This file is license under the terms of a Creative Commons Attribution 
% 3.0 Unported License, or (at your option) any later version. More details
% at <http://creativecommons.org/licenses/by/3.0/>.
\begin{tabular}{cc|cc|cc}
    \hline
    Com. & Res. & Com. & Res. & Com. & Res. \\ \hline
    \lstinline!+! & $+$ & \lstinline!-! & $-$ & \lstinline!\pm! & $\pm$ \\
    \lstinline!\mp! & $\mp$ & \lstinline!\times! & $\times$ & \lstinline!\cdot! & $\cdot$ \\
    \lstinline!\div! & $\div$ & \lstinline!\And! & $\And$ & \lstinline!\setminus! & $\setminus$ \\
    \lstinline!\smallsetminus! & $\smallsetminus$ & \lstinline!\dagger! & $\dagger$ & \lstinline!\ddagger! & $\ddagger$ \\
    \lstinline!\ast! & $\ast$ & \lstinline!\star! & $\star$ & \lstinline!\wedge! & $\wedge$ \\
    \lstinline!\vee! & $\vee$ & \lstinline!\cap! & $\cap$ & \lstinline!\cup! & $\cup$ \\
    \lstinline!\sqcap! & $\sqcap$ & \lstinline!\sqcup! & $\sqcup$ & \lstinline!\oplus! & $\oplus$ \\
    \lstinline!\ominus! & $\ominus$ & \lstinline!\otimes! & $\otimes$ & \lstinline!\oslash! & $\oslash$ \\
    \lstinline!\odot! & $\odot$ & \lstinline!\bigcirc! & $\bigcirc$ & \lstinline!\circ! & $\circ$ \\
    \lstinline!\bullet! & $\bullet$ & \lstinline!\bigtriangleup! & $\bigtriangleup$ & \lstinline!\bigtriangledown! & $\bigtriangledown$ \\
    \lstinline!\triangleleft! & $\triangleleft$ & \lstinline!\triangleright! & $\triangleright$ &\lstinline!\diamond! & $\diamond$ \\
    \lstinline!\wr! & $\wr$ & \lstinline!\amalg! & $\amalg$ \\ \hline
\end{tabular}

\end{table}
\begin{table}[!htb]
  \centering
  \caption{Operadores puros.}
  \label{tab:math_functions1}
  % Filename: math_functions1@latex_with_vim.tex
% This code is part of LaTeX with Vim.
% 
% Description: LaTeX with Vim is free book about Vim, LaTeX and Git.
% 
% Created: 30.03.12 12:16:46 AM
% Last Change: 30.03.12 12:16:51 AM
% 
% Author: Raniere Gaia Costa da Silva, r.gaia.cs@gmail.com
% Organization:  
% 
% Copyright (c) 2010, 2011, 2012, Raniere Gaia Costa da Silva. All rights 
% reserved.
% 
% This file is license under the terms of a Creative Commons Attribution 
% 3.0 Unported License, or (at your option) any later version. More details
% at <http://creativecommons.org/licenses/by/3.0/>.
\begin{tabular}{cc|cc|cc}
    \hline
    Com. & Res. & Com. & Res. & Com. & Res. \\ \hline
    \lstinline!\log! & $\log$ & \lstinline!\ln! & $\ln$ & \lstinline!\exp! & $\exp$ \\
    \lstinline!\arccos! & $\arccos$ & \lstinline!\arcsin! & $\arcsin$ & \lstinline!\arctan! & $\arctan$ \\
    \lstinline!\cos! & $\cos$ & \lstinline!\sin! & $\sin$ & \lstinline!\tan! & $\tan$ \\
    \lstinline!\csc! & $\csc$ & \lstinline!\sec! & $\sec$ & \lstinline!\cot! & $\cot$ \\
    \lstinline!\cosh! & $\cosh$ & \lstinline!\sinh! & $\sinh$ & \lstinline!\tanh! & $\tanh$ \\
    \lstinline!\lg! & $\lg$ & \lstinline!\arg! & $\arg$ & \lstinline!\hom! & $\hom$ \\
    \lstinline!\dim! & $\dim$ & \lstinline!\ker! & $\ker$ & \lstinline!\det! & $\det$ \\
    \lstinline!\gcd! & $\gcd$ & & & & \\ \hline
\end{tabular}

\end{table}
\begin{table}[!htb]
  \centering
  \caption{Operadores com intervalos.}
  \label{tab:math_functions2}
  % Filename: math_functions2@latex_with_vim.tex
% This code is part of LaTeX with Vim.
% 
% Description: LaTeX with Vim is free book about Vim, LaTeX and Git.
% 
% Created: 30.03.12 12:17:02 AM
% Last Change: 30.03.12 12:17:06 AM
% 
% Author: Raniere Gaia Costa da Silva, r.gaia.cs@gmail.com
% Organization:  
% 
% Copyright (c) 2010, 2011, 2012, Raniere Gaia Costa da Silva. All rights 
% reserved.
% 
% This file is license under the terms of a Creative Commons Attribution 
% 3.0 Unported License, or (at your option) any later version. More details
% at <http://creativecommons.org/licenses/by/3.0/>.
\begin{tabular}{cc|cc|cc|cc}
    \hline
    Comando & Resultado & Comando & Resultado & Comando & Resultado & Comando & Resultado \\ \hline
    \textbackslash\textsf{int} & $\int$ & \textbackslash\textsf{iint} & $\iint$ & \textbackslash\textsf{iiint} & $\iiint$ & \textbackslash\textsf{iiiint} & $\iiiint$ \\
    \textbackslash\textsf{idotsint} & $\idotsint$ & \textbackslash\textsf{oint} & $\oint$ & \textbackslash\textsf{prod} & $\prod$ & \textbackslash\textsf{coprod} & $\coprod$ \\ 
    \textbackslash\textsf{bigcap} & $\bigcap$ & \textbackslash\textsf{bigcup} & $\bigcup$ & \textbackslash\textsf{bigwedge} & $\bigwedge$ & \textbackslash\textsf{bigvee} & $\bigvee$ \\ 
    \textbackslash\textsf{bigsqcup} & $\bigsqcup$ & \textbackslash\textsf{biguplus} & $\biguplus$ & \textbackslash\textsf{bigotimes} & $\bigotimes$ & \textbackslash\textsf{bigoplus} & $\bigoplus$ \\ 
    \textbackslash\textsf{bigodot} & $\bigodot$ & \textbackslash\textsf{sum} & $\sum$ & & & & \\ \hline
\end{tabular}

\end{table}
\begin{table}[!htb]
  \centering
  \caption{Operadores similares ao limites.}
  \label{tab:math_functions3}
  % Filename: math_functions3@latex_with_vim.tex
% This code is part of LaTeX with Vim.
% 
% Description: LaTeX with Vim is free book about Vim, LaTeX and Git.
% 
% Created: 30.03.12 12:17:16 AM
% Last Change: 30.03.12 12:17:21 AM
% 
% Author: Raniere Gaia Costa da Silva, r.gaia.cs@gmail.com
% Organization:  
% 
% Copyright (c) 2010, 2011, 2012, Raniere Gaia Costa da Silva. All rights 
% reserved.
% 
% This file is license under the terms of a Creative Commons Attribution 
% 3.0 Unported License, or (at your option) any later version. More details
% at <http://creativecommons.org/licenses/by/3.0/>.
\begin{tabular}{cc|cc|cc|cc}
    \hline
    Comando & Resultado & Comando & Resultado & Comando & Resultado \\ \hline
    \textbackslash\textsf{lim} & $\lim$ & \textbackslash\textsf{inf} & $\inf$ & \textbackslash\textsf{sup} & $\sup$ & \textbackslash\textsf{max} & $\max$ \\
    \textbackslash\textsf{injlim} & $\injlim$ & \textbackslash\textsf{liminf} & $\liminf$ & \textbackslash\textsf{limsup} & $limsup$ & \textbackslash\textsf{min} & $\min$ \\
    \textbackslash\textsf{varinjlim} & $\varinjlim$ & \textbackslash\textsf{varliminf} & $\varliminf$ & \textbackslash\textsf{varlimsup} & $varlimsup$ & \textbackslash\textsf{Pr} & $\Pr$ \\
    \textbackslash\textsf{projlim} & $\projlim$ & \textbackslash\textsf{varprojlim} & $\varprojlim$ & \\ \hline
\end{tabular}

\end{table}
\begin{table}[!htb]
  \centering
  \caption{Outros símbolos matemáticos}
  \label{tab:math_others}
  % Filename: math_others@latex_with_vim.tex
% This code is part of LaTeX with Vim.
% 
% Description: LaTeX with Vim is free book about Vim, LaTeX and Git.
% 
% Created: 30.03.12 12:18:26 AM
% Last Change: 30.03.12 12:18:30 AM
% 
% Author: Raniere Gaia Costa da Silva, r.gaia.cs@gmail.com
% Organization:  
% 
% Copyright (c) 2010, 2011, 2012, Raniere Gaia Costa da Silva. All rights 
% reserved.
% 
% This file is license under the terms of a Creative Commons Attribution 
% 3.0 Unported License, or (at your option) any later version. More details
% at <http://creativecommons.org/licenses/by/3.0/>.
\begin{tabular}{cc|cc|cc}
    \hline
    Com. & Res. & Com. & Res. & Com. & Res. \\ \hline
    \lstinline!\Re! & $\Re$ & \lstinline!\Im! & $\Im$ & \lstinline!\nabla! & $\nabla$ \\
    \lstinline!\partial! & $\partial$ & \lstinline!\infty! & $\infty$ & \lstinline!\emptyset! & $\emptyset$ \\
    \lstinline!\varnothing! & $\varnothing$ & \lstinline!\forall! & $\forall$ & \lstinline!\exists! & $\exists$ \\
    \lstinline!\nexists! & $\nexists$ & \lstinline!\angle! & $\angle$ & \lstinline!\measuredangle! & $\measuredangle$ \\
    \lstinline!\sphericalangle! & $\sphericalangle$ & \lstinline!\top! & $\top$ & \lstinline!\bot! & $\bot$ \\
    \lstinline!\diagup! & $\diagup$ & \lstinline!\diagdown! & $\diagdown$ & \lstinline!\triangle! & $\triangle$ \\
    \lstinline!\triangledown! & $\triangledown$ & \lstinline!\blacktriangle! & $\blacktriangle$ & \lstinline!\blacktriangledown! & $\blacktriangledown$ \\
    \lstinline!\Diamond! & $\Diamond$ & \lstinline!\lozenge! & $\lozenge$ & \lstinline!\blacklozenge! & $\blacklozenge$ \\
    \lstinline!\bigstar! & $\bigstar$ & \lstinline!\Box! & $\Box$ & \lstinline!\square! & $\square$ \\
    \lstinline!\blacksquare! & $\blacksquare$ & \lstinline!\clubsuit! & $\clubsuit$ & \lstinline!\diamondsuit! & $\diamondsuit$ \\
    \lstinline!\heartsuit! & $\heartsuit$ & \lstinline!\spadesuit! & $\spadesuit$ \\ \hline
\end{tabular}

\end{table}
\begin{table}[h!tb]
  \centering
  \caption{Alfabeto Grego, letras minúsculas}
  \label{tab:math_greek}
  % Filename: math_greek@latex_with_vim.tex
% This code is part of LaTeX with Vim.
% 
% Description: LaTeX with Vim is free book about Vim, LaTeX and Git.
% 
% Created: 30.03.12 12:17:35 AM
% Last Change: 30.03.12 12:17:39 AM
% 
% Author: Raniere Gaia Costa da Silva, r.gaia.cs@gmail.com
% Organization:  
% 
% Copyright (c) 2010, 2011, 2012, Raniere Gaia Costa da Silva. All rights 
% reserved.
% 
% This file is license under the terms of a Creative Commons Attribution 
% 3.0 Unported License, or (at your option) any later version. More details
% at <http://creativecommons.org/licenses/by/3.0/>.
\begin{tabular}{cc|cc|cc|cc}
    \hline
    Com. & Res. & Com. & Res. & Com. & Res. & Com. & Res. \\ \hline
    \lstinline!\alpha! & $\alpha$ & \lstinline!\beta! & $\beta$ & \lstinline!\gamma! & $\gamma$ & \lstinline!\delta! & $\delta$ \\
    \lstinline!\epsilon! & $\epsilon$ & \lstinline!\zeta! & $\zeta$ & \lstinline!\eta! & $\eta$ & \lstinline!\theta! & $\theta$ \\
    \lstinline!\iota! & $\iota$ & \lstinline!\kappa! & $\kappa$ & \lstinline!\lambda! & $\lambda$ & \lstinline!\mu! & $\mu$ \\
    \lstinline!\nu! & $\nu$ & \lstinline!\xi! & $\xi$ & \lstinline!\pi! & $\pi$ & \lstinline!\rho! & $\rho$ \\
    \lstinline!\sigma! & $\sigma$ & \lstinline!\tau! & $\tau$ & \lstinline!\upsilon! & $\upsilon$ & \lstinline!\phi! & $\phi$ \\
    \lstinline!\chi! & $\chi$ & \lstinline!\psi! & $\psi$ & \lstinline!\omega! & $\omega$ & \lstinline!\digamma! & $\digamma$ \\
    \lstinline!\varepsilon! & $\varepsilon$ & \lstinline!\vartheta! & $\vartheta$ & \lstinline!\varkappa! & $\varkappa$ & \lstinline!\varpi! & $\varpi$ \\
    \lstinline!\varrho! & $\varrho$ & \lstinline!\varsigma! & $\varsigma$ & \lstinline!\varphi! & $\varphi$ & & \\ \hline
\end{tabular}

\end{table}
\begin{table}[!tb]
  \centering
  \caption{Alfabeto Grego, letras maiúsculo}
  \label{tab:math_greek_capital}
  % Filename: math_greek_capital@latex_with_vim.tex
% This code is part of LaTeX with Vim.
% 
% Description: LaTeX with Vim is free book about Vim, LaTeX and Git.
% 
% Created: 30.03.12 12:17:50 AM
% Last Change: 30.03.12 12:17:56 AM
% 
% Author: Raniere Gaia Costa da Silva, r.gaia.cs@gmail.com
% Organization:  
% 
% Copyright (c) 2010, 2011, 2012, Raniere Gaia Costa da Silva. All rights 
% reserved.
% 
% This file is license under the terms of a Creative Commons Attribution 
% 3.0 Unported License, or (at your option) any later version. More details
% at <http://creativecommons.org/licenses/by/3.0/>.
\begin{tabular}{cc|cc|cc|cc}
    \hline
    Com. & Res. & Com. & Res. & Com. & Res. & Com. & Res. \\ \hline
    \lstinline!\Gamma! & $\Gamma$ & \lstinline!\Delta! & $\Delta$ & \lstinline!\Theta! & $\Theta$ & \lstinline!\Lambda! & $\Lambda$ \\
    \lstinline!\Xi! & $\Xi$ & \lstinline!\Pi! & $\Pi$ & \lstinline!\Sigma! & $\Sigma$ & \lstinline!\Upsilon! & $\Upsilon$ \\
    \lstinline!\Phi! & $\Phi$ & \lstinline!\Psi! & $\Psi$ & \lstinline!\Omega! & $\Omega$ \\
    \lstinline!\varGamma! & $\varGamma$ & \lstinline!\varDelta! & $\varDelta$ & \lstinline!\varTheta! & $\varTheta$ & \lstinline!\varLambda! & $\varLambda$ \\
    \lstinline!\varXi! & $\varXi$ & \lstinline!\varPi! & $\varPi$ & \lstinline!\varSigma! & $\varSigma$ & \lstinline!\varUpsilon! & $\varUpsilon$ \\
    \lstinline!\varPhi! & $\varPhi$ & \lstinline!\varPsi! & $\varPsi$ &
    \lstinline!\varOmega! & $\varOmega$ & & \\ \hline
\end{tabular}

\end{table}

\subsection{Raiz quadrada}
Utiliza-se o comando \lstinline!\sqrt! para raiz quadrada\index{modo
matematico@modo matemático!raiz quadrada}. \\
\example{codes/math_sqrt@latex_with_vim.tex}

\subsection{Binomial}
Utiliza-se o comando \lstinline!\binom! para os binômios\index{modo
matematico@modo matemático!binomio@binômio}. \\
\example{codes/math_binom@latex_with_vim.tex}

\subsection{Congruências}
A forma mais comum para congruências\index{modo matematico@modo
matemático!congruencia@congruência} corresponde ao uso dos comandos
\lstinline!\equiv! e \lstinline!\pmod!. \\
\example{codes/math_mod@latex_with_vim.tex}
