\chapter{Olá \LaTeX} \label{sch:basic}
Neste primeiro capítulo apresentamos os conhecimentos mínimos de todo usuário
do LaTeX.

\section{Instalação}\index{instalacao@instalação}
Para utilizar o LaTeX você precisa das macros que compõem o mesmo. A forma mais
fácil de conseguir isso é instalando uma distribuição da lista abaixo:
\begin{itemize}
  \item Linux: TeX Live\index{TeX Live|see{instalação}} (\url{http://www.tug.org/texlive}),
  \item Mac OS X: TeX Live\index{TeX Live|see{instalação}}
    (\url{http://www.tug.org/texlive}), MacTeX\index{Mac OS X|see{instalação}}
    (\url{http://www.tug.org/mactex/}),
  \item Windows: TeX Live\index{TeX Live|see{instalação}}
    (\url{http://www.tug.org/texlive}), proTeXt\index{proTeXt|see{instalação}}
    (\url{http://www.tug.org/protext/}) ou MiKTeX\index{MikTeX|see{instalação}}
    (\url{http://www.miktex.org/}).
\end{itemize}
Além das macros também é necessário um editor de texto ou uma IDE\index{IDE}
(\flang{Integrated Development Environment}) própria para o LaTeX, como
\begin{itemize}
  \item GNU Emacs\index{Emacs|see{IDE}}
    (\url{http://www.gnu.org/software/emacs/}) com o AUCTeX
    (\url{http://www.gnu.org/software/auctex/}),
  \item TeXworks\index{TeXworks|see{IDE}} (\url{http://www.leliseron.org/texworks/}),
  \item Kile\index{Kile|see{IDE}} (\url{http://kile.sourceforge.net/}),
  \item Texmaker\index{Texmaker|see{IDE}} (\url{http://www.xm1math.net/texmaker/}).
\end{itemize}
Uma lista com várias IDE's encontra-se disponível em
\url{http://en.wikipedia.org/wiki/Comparison_of_TeX_editors}\nocite{Wikipedia:EN:Comparison_TeX_editors}.

\section{Arquivo \lcode{.tex}} \label{sse:basic:tex}
O LaTeX utiliza \lcode{.tex}\index{.tex@\lcode{.tex}} como extensão padrão. O
arquivo \lcode{main.tex}, onde \lcode{main} representa o nome do arquivo
\lcode{.tex}, é um arquivo de texto, estruturado em duas partes:
\begin{enumerate}
  \item \emph{preâmbulo}\index{preambulo@\emph{preâmbulo}}
  \item \emph{informação}\index{informacao@\emph{informação}}
\end{enumerate}
sendo que a segunda parte deve ser delimitada pelo ambiente \envname{document},
i.e., ser incluída no lugar de \lcode{XXX} do código abaixo:
\begin{code}
  \begin{document}
  XXX
  \end{document}
\end{code}

É permito incluir um ou mais arquivo dentro de \lcode{main.tex}, isto é,
trabalhar com múltiplos arquivos\index{multiplos arquivos@múltiplos arquivos}.
Os arquivos a serem incluídos também possuem a extensão \lcode{.tex} mas devem
conter apenas a \emph{informação}.\footnote{Ao trabalhar com múltiplos arquivos
deve-se apenas compilar o arquivo \lcode{main.tex}.}

Uma das forma de incluir um arquivo é com o comando
\lstinline!\input!\index{comando!input@\lstinline+\input+}, como ilustrado a
seguir:
\begin{code}
  \input{aux.tex}
\end{code}
onde \lcode{aux.tex} é o nome do arquivo a ser incluído.\footnote{Caso a
extensão do arquivo seja suprimida será utilizada \lcode{.tex}.}

Quando \lcode{main.tex} for compilado o arquivo \lcode{aux.tex} será lido e
processado exatamente como se tive-se sido inserido na posição que o comando
\lstinline!\input! ocupa.

\section{\emph{Preâmbulo}} \label{sse:basic:preamble}
O \emph{preâmbulo}\index{preambulo@\emph{preâmbulo}} deve ser iniciado por
\begin{code}
  \documentclass[opcoes]{classe}
\end{code}
onde \lcode{classe}\index{comando!documentclass@\lstinline+\documentclass+}
indica o tipo de documento a ser criado e \lcode{opcoes} é uma lista de
palavras chaves separadas por vírgula que personaliza o comportamento de
\lcode{classe} (na Tabela \ref{tab:par_options} encontra-se algumas das palavras
chaves disponíveis).
\begin{table}[h!tb]
  \centering
  \caption{Parâmetros disponíveis para \lcode{opcoes}.} \label{tab:par_options}
  \begin{tabular}{llp{0.6\textwidth}}
    \hline
    Função & Código & Descrição \\ \hline
    \multirow{4}{*}{Tamanho} &  & Utiliza, por padrão, o tamanho 10. \\
    & \lcode{10pt} & Tamanho 10. \\
    & \lcode{11pt} & Tamanho 11. \\
    & \lcode{12pt} & Tamanho 12. \\ \hline
    \multirow{7}{*}{Papel} & & Utiliza, por padrão, o tamanho da folha correspondente carta. \\
    & \lcode{letterpaper} & Tamanho da folha correspondente carta. \\
    & \lcode{a4paper} & Tamanho da folha correspondente a A4. \\
    & \lcode{a5paper} & Tamanho da folha correspondente a A5. \\
    & \lcode{b5paper} & Tamanho da folha correspondente a B5. \\
    & \lcode{executivepaper} & Tamanho da folha correspondente a folha executiva. \\
    & \lcode{legalpaper} & Tamanho da folha correspondente a folha legal. \\ \hline
    \multirow{2}{*}{Al. equação} & & Por padrão centra as equações. \\
    & \lcode{fleqn} & Alinha as equações à esquerda. \\ \hline
    \multirow{2}{*}{Nº equação} & & Por padrão enumera as equações à direita. \\
    & \lcode{leqno} & Enumera as equações à esquerda. \\ \hline
    \multirow{4}{*}{Título} & & Por padrão a classe \lcode{article} não começa uma nova página após o título, enquanto que \lcode{report} e \lcode{book} o fazem. \\
    & \lcode{titlepage} & Começa uma nova página após o título. \\
    & \lcode{leqno} & Não começa uma nova página após o título. \\ \hline
    \multirow{4}{*}{Faces} & & Por padrão a classe \lcode{article} e \lcode{report} são a uma face e a classe \lcode{book} é a duas. \\
    & \lcode{oneside} & Gera o documento a uma face. \\
    & \lcode{twoside} & Gera o documento a duas fazes. \\ \hline
    \multirow{5}{*}{Começo} & & Não funciona com a classe \lcode{article} por nesta não existirem capítulos e por padrão a classe \lcode{report} começa os capítulos na próxima página disponível e a classe \lcode{book} sempre nas páginas à direita. \\
    & \lcode{openright} & Começa os capítulos sempre nas páginas à direita. \\
    & \lcode{openany} & Começa os capítulos na próxima página disponível. \\ \hline
    Colunas & \lcode{twocolumn} & Gera o arquivo utilizando-se de duas colunas. \\
    \hline
\end{tabular}

\end{table}

\lcode{class}\index{comando!documentclass@\lstinline+\documentclass+!class@\lcode{class}}
corresponde ao nome de um arquivo \lcode{.cls}, os principais são apresentados
na Tabela \ref{tab:documentclass} e outros são indicados em
\url{http://aprendolatex.wordpress.com/2007/07/15/mais-classes-de-documentos/}.
Existe ainda alguns arquivos \lcode{.cls} personalizados disponíveis na
internet, destacando-se o \lcode{abnt.cls}, disponível em
\url{http://abntex.codigolivre.org.br/}, indicado para documentos que devem
seguir as normas da ABNT e o usuário também pode escrever sua própria
\lcode{classe}.
\begin{table}[h!tb]
  \centering
  \caption{Parâmetros disponíveis para \lcode{classe}.} \label{tab:documentclass}
  % Filename: documentclass@latex_with_vim.tex
% This code is part of LaTeX with Vim.
% 
% Description: LaTeX with Vim is free book about Vim, LaTeX and Git.
% 
% Created: 30.03.12 12:12:55 AM
% Last Change: 30.03.12 12:13:08 AM
% 
% Author: Raniere Gaia Costa da Silva, r.gaia.cs@gmail.com
% Organization:  
% 
% Copyright (c) 2010, 2011, 2012, Raniere Gaia Costa da Silva. All rights 
% reserved.
% 
% This file is license under the terms of a Creative Commons Attribution 
% 3.0 Unported License, or (at your option) any later version. More details
% at <http://creativecommons.org/licenses/by/3.0/>.
\begin{tabular}{lp{0.8\textwidth}}
    \hline
    Código & Descrição \\ \hline
    \lcode{article} & Para artigos em revistas especializadas, palestras, trabalhos de disciplinas \dots \\
    \lcode{report} & Para informes maiores que constam de mais de um capítulo, projetos de fim de curso, dissertações, teses e similares. \\
    \lcode{book} & Para livros. \\
    \lcode{slide} & Para transparências. \\
    \lcode{beamer} & Para apresenta\c{c}\~{o}es. \\
    \lcode{exam} & Para lista de exerc\'{i}cios. \\
    \hline
\end{tabular}

\end{table}

\section{Hello world} \label{sse:basic:hello_world}
Anteriormente foi apresentado os aplicativos necessários para trabalhar com
LaTeX e as duas partes principais do arquivo \lcode{.tex}. A seguir
apresentaremos como construir a
\emph{informação}\index{informacao@\emph{informação}}.

O documento mais simples que podemos criar é apresentado abaixo. \\ 
\begin{minipage}[t]{0.5\linewidth}
  \vspace{-8pt}
  \begin{code}
    \documentclass[10pt,a4paper]{article}
    \begin{document}
    Hello world.
    \end{document}
  \end{code}
\end{minipage} \quad \vrule \quad
\begin{minipage}[t]{0.35\linewidth} \vspace{0pt}
  Hello world.
\end{minipage}

Os exemplos que serão apresentados aparecerão seguindo o modelo acima, isto é,
em duas colunas sendo a coluna da esquerda contendo o código LaTeX e a coluna da
direita contendo a saída obtida. Por simplicidade, nos demais exemplos iremos
apresentar apenas a \emph{informação}.

\subsection{Espaços, linhas, parágrafos e páginas} \label{sss:basic:space}
No LaTeX o espaço entre palavras apresenta uma particularidade: ele é ignorado
se houver dois ou mais espaços seguidos, como podemos observar a seguir. \\
\example{codes/hello_spaces@latex_with_vim.tex}

Quando for necessário gerar dois ou mais espaços seguidos deve-se utilizar a
barra invertida entre os espaços como ilustrado a seguir. \\
\example{codes/hello_spaces_backslash@latex_with_vim.tex}

Nos dois exemplos anteriores é possível verificar que a mudança de linha no
código não produz uma nova linha\index{nova linha} no documento gerado. A
quebra de linha no LaTeX é representada por \lstinline!\\!\index{comando!
@\lstinline+\\+} ou pelo comando
\lstinline!\newline!\index{comando!newline@\lstinline+\newline+}, como ilustrada
a seguir. \\
\example{codes/hello_newlines@latex_with_vim.tex}

Já a mudança de parágrafo\index{paragrafo@parágrafo} é indicada por uma linha em
branco. 

Quando for necessário forçar uma mudança de página utiliza-se o comando
\lstinline!\newpage!\index{comando!newpage@\lstinline+\newpage+}. Assim como o
LaTeX ignora dois ou mais espaços seguidos a mudança de linha e de página também
é ignorada.

\subsection{Hifenização}
O LaTeX tenta balancear o tamanho das linhas a serem geradas e para isso
utiliza-se de um banco de dados para hifenizar, quando necessário, alguma
palavra.

Algumas vezes a hifenização\index{hifenizacao@hifenização} ocorre de maneira
inadequada e para corrigir devemos utilizar o comando
\lstinline!\hyphenation!\index{comando!hyphenation@\lstinline+\hyphenation+}
cujo parâmetro é uma lista de palavras, separadas por espaço, onde o comando - é
utilizado para indicar onde a palavra pode ser separada.

\subsection{Acentos}
Para inserir os acentos deve-se utilizar a codificação presente na
Tabela~\ref{tab:diacritic}.
\begin{table}[!htb]
  \centering
  \caption{Acentuação (utilizando a vogal ``o'' para exemplo).} \label{tab:diacritic}
  % Filename: diacrict@latex_with_vim.tex
% This code is part of LaTeX with Vim.
% 
% Description: LaTeX with Vim is free book about Vim, LaTeX and Git.
% 
% Created: 30.03.12 12:12:36 AM
% Last Change: 30.03.12 12:12:40 AM
% 
% Author: Raniere Gaia Costa da Silva, r.gaia.cs@gmail.com
% Organization:  
% 
% Copyright (c) 2010, 2011, 2012, Raniere Gaia Costa da Silva. All rights 
% reserved.
% 
% This file is license under the terms of a Creative Commons Attribution 
% 3.0 Unported License, or (at your option) any later version. More details
% at <http://creativecommons.org/licenses/by/3.0/>.
\begin{tabular}{cc|cc|cc|cc}
    \hline
    Comando & Resultado & Comando & Resultado & Comando & Resultado & Comando & Resultado \\ \hline
    \textbackslash '\{o\} & \'{o} & \textbackslash =\{o\} & \={o} & \textbackslash u\{o\} & \u{o} & \textbackslash .\{o\} & \.{o} \\
    \textbackslash v\{o\} & \v{o} & \textbackslash r\{o\} & \r{o} & \textbackslash c\{c\} & \c{c} & \textbackslash t\{oo\} & \t{oo} \\
    \textbackslash \textasciicircum \{o\} & \^{o} & \textbackslash \textasciitilde \{o\} & \~{o} & \textbackslash "\{o\} & \"{o} & \textbackslash d\{o\} & \d{o} \\
    \textbackslash H\{o\} & \H{o} & \textbackslash b\{o\} & \b{o} & \textbackslash `\{o\} & \`{o} & \textbackslash i & \i \\ \hline
\end{tabular}

\end{table}

\section{Caracteres especiais}
No LaTeX alguns caracteres apresentam forma própria de representação. A seguir
enunciaremos alguns.

\subsection{Aspas}
Para as aspas\index{aspas} não deve-se usar o caractere de aspas. Para abrir as
aspas deve-se utilizar o acento simples e para fechar a aspa simples. \\
\example{codes/quotation_mark@latex_with_vim.tex}

\subsection{Traço}
LaTeX admite três tipos de traço\index{traco@traço}. \\
\example{codes/dashes@latex_with_vim.tex}

\subsection{Pontos sucessivos}
Utiliza-se o comando \lstinline!\dots! ou \lstinline!\ldots! para pontos
sucessivos. \\
\example{codes/dots@latex_with_vim.tex}

\subsection{Pontuação e demais símbolos}
Para pontuação\index{pontuacao@pontuação} e demais símbolos especias deve-se
proceder como na Tabela~\ref{tab:symbols}.
\begin{table}[h!tb]
  \centering
  \caption{Para pontuação e símbolos especias.}
  \label{tab:symbols}
  % File: symbols@latex-with-vim.tex
% This code is part of LaTeX with Vim.
% 
% Description: LaTeX with Vim is free book about Vim, LaTeX and Git.
% 
% Created: 30.03.12 12:19:38 AM
% Last Change: 30.03.12 12:19:44 AM
% 
% Author: Raniere Gaia Costa da Silva, r.gaia.cs@gmail.com
% Organization:  
% 
% Copyright (c) 2010, 2011, 2012, Raniere Gaia Costa da Silva. All rights 
% reserved.
% 
% This file is license under the terms of a Creative Commons Attribution 
% 3.0 Unported License, or (at your option) any later version. More details
% at <http://creativecommons.org/licenses/by/3.0/>.

\begin{tabular}{cc|cc|cc}
    \hline
    Comando & Resultado & Comando & Resultado & Comando & Resultado \\ \hline
    \textbackslash \& & \& & \textbackslash textasteriskcentered & \textasteriskcentered & \textbackslash textbackslash & \textbackslash \\
    \textbackslash textbar & \textbar & \textbackslash \{ & \{ & \textbackslash \} & \} \\
    \textbackslash texbullet & \textbullet & \textbackslash textasciitilde & \textasciitilde & \textbackslash textasciicircum & \textasciicircum \\
    \textbackslash copyright & \copyright & \textbackslash textregistered & \textregistered & \textbackslash texttrademark & \texttrademark \\
    \textbackslash textperiodcentered & \textperiodcentered & \textbackslash textexclamdown & \textexclamdown & \textbackslash textquestiondown & \textquestiondown \\
    \textbackslash \% & \% & \textbackslash textgreater & \textgreater & \textbackslash textless & \textless  \\
    \textbackslash \# & \# & \textbackslash S & \S & \textbackslash P & \P \\
    \textbackslash \_ & \_ & \textbackslash dag & \dag & \textbackslash ddag & \ddag \\
    \textbackslash pounds & \pounds & \textbackslash textsuperscript\{a\} & \textsuperscript{a} & \textbackslash textcircled\{a\} & \textcircled{a} \\
    \textbackslash textvisiblespace & \textvisiblespace & \textbackslash \$ & \$ & \textbackslash euro & \euro \\ \hline
\end{tabular}

\end{table}

\subsection{Comentários}
Também é possível inserir comentários\index{comentarios@comentários} no arquivo
\lcode{.tex}, utilizando-se para isso do caractere \lstinline!%!\index{comando!
@\lstinline+%+} de forma que todo o texto posterior ao mesmo e na mesma linha é
considerado comentário e consequentemente ignorado pelo compilador.

\section{Apresentações}
Apresentações podem ser criadas com a classe \lstinline{beamer} e organizadas pelo ambiente \envname{frame}\index{ambiente!frame@\envname{frame}} que delimita onde começa e termina cada um dos \flang{slides} da apresentação. A seguir apresentamos uma apresentação bem simples para exemplificar a utilização do ambiente \envname{frame}. \\
\examplebeamer{codes/beamer_minimal@latex_with_vim}

\subsection{Primeiro \flang{slide}}
Para a criação do primeiro \flang{slide} com o título e autor pode utilizar os
comandos \lstinline!\title! e \lstinline!\author! e, delimitado pelo ambiente
\envname{frame}, o comando \lstinline!\titlepage!.

Além dos comandos \lstinline!\title! e \lstinline!\author! estão disponíveis os
comandos \lstinline!\subtitle!, \lstinline!\date! e \lstinline!\institute! que
correspondem, respectivamente, ao subtítulo, data e local em que a apresentação
irá ocorrer. Exceto pelo comando \lstinline!\date! todos os demais comandos
aceitam como opção uma abreviação do parâmetro. \\
\examplebeamer{codes/beamer_first_page@latex_with_vim}

\subsection{Título do \flang{slide}}
Para cada \flang{slide} é possível atribuir um título com o comando
\lstinline!\frametitle! que normalmente será apresentado no topo do
\flang{slide}. \\
\examplebeamer{codes/beamer_title@latex_with_vim}

\subsection{Comandos e ambientes do LaTeX}
A classe \pkgname{beamer} é compatível com grande parte dos comandos e ambientes
do LaTeX sejam estes nativos ou presentes em algum pacote, i.e., para incluir
listas, figuras, tabelas, expressões matemáticas, \ldots utiliza-se os mesmos
comandos e ambientes. \\
\examplebeamer{codes/beamer_enumerate@latex_with_vim} \\
\examplebeamer{codes/beamer_math@latex_with_vim}

\subsection{\flang{Overlays}}\index{beamer!overlay@\flang{overlay}}
Até o momento todos os \flang{slides} que construímos tinha sua informação
apresentada em um único momento. Infelizmente não é isso que deseja-se na grande
maioria da apresentações, i.e., deseja-se que fragmentos dos \flang{slides}
sejam apresentados em momentos distintos para que seja possível construir a
informação desejada.

Para fragmentar o conteúdo dos \flang{slides} podemos utilizar o comando
\lstinline!\pause!\index{comando!pause@\lstinline+\pause+} na posição que
deseja-se fragmentar os \flang{slides}. \\
\begin{minipage}[c]{0.5\textwidth}
    \fcode{codes/beamer_overlays01@latex_with_vim.tex}
\end{minipage} \quad \vrule \quad
\begin{minipage}[c]{0.35\textwidth}
    \fbox{\includegraphics[width=\textwidth, page=1]{codes/beamer_overlays01@latex_with_vim.pdf}}
    \fbox{\includegraphics[width=\textwidth, page=2]{codes/beamer_overlays01@latex_with_vim.pdf}}
\end{minipage}

O comando \lstinline!\pause! funciona dentro de vários ambientes do LaTeX sejam
estes nativos ou presentes em algum pacote. No exemplo a seguir utilizamos o
comando \lstinline!\pause! dentro do ambiente \envname{tikzpicture}. \\
\begin{minipage}[c]{0.5\textwidth}
    \fcode{codes/beamer_overlays02@latex_with_vim.tex}
\end{minipage} \quad \vrule \quad
\begin{minipage}[c]{0.35\textwidth}
    \fbox{\includegraphics[width=\textwidth, page=1]{codes/beamer_overlays02@latex_with_vim.pdf}}
    \fbox{\includegraphics[width=\textwidth, page=2]{codes/beamer_overlays02@latex_with_vim.pdf}}
\end{minipage}

\subsection{Temas}\index{beamer!tema}
Até o momento, os \flang{slides} apresentados possuiam fundo e bordas muito
simples. É possível mudar isso utilizando os comandos
\lstinline!\usecolortheme!, muda apenas o esquema de cores, e
\lstinline!\usetheme!, mais genérico. \\
\examplebeamer{codes/beamer_usetheme01@latex_with_vim} \\
\examplebeamer{codes/beamer_usetheme02@latex_with_vim}

Para conhecer algumas dos parâmetros disponíveis para os comandos
\lstinline!usecolortheme! e \lstinline!\usetheme! sugere-se
\url{http://www.hartwork.org/beamer-theme-matrix/}. Outros temas estão
disponíveis na internet e alguns deles reunidos em
\url{http://latex.simon04.net/}.
