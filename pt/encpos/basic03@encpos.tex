\chapter{Um pouco de layout}
Enquanto que no capítulo anterior foi apresentado algumas ferramentas para
escrever textos mais complexos, por exemplo, contendo listas e tabelas, nesse
capítulo iremos tratar um pouco do layout do texto.

\section{Fonte}
No LaTeX estão disponíveis algumas fontes\index{fonte} opcionais. Comandos da
forma \lstinline!\textXX! são responsáveis por alterar a fonte sendo que
\lcode{XX} corresponde ao código da fonte a serem utilizados. A
Tabela~\ref{tab:text} apresenta alguns das opções disponíveis.
\begin{table}[!htb]
  \centering
  \caption{Opções disponíveis para \lcode{XX} da fonte.} \label{tab:text}
  % Filename: text@latex_with_vim.tex
% This code is part of LaTeX with Vim.
% 
% Description: LaTeX with Vim is free book about Vim, LaTeX and Git.
% 
% Created: 30.03.12 12:19:02 AM
% Last Change: 30.03.12 12:19:06 AM
% 
% Author: Raniere Gaia Costa da Silva, r.gaia.cs@gmail.com
% Organization:  
% 
% Copyright (c) 2010, 2011, 2012, Raniere Gaia Costa da Silva. All rights 
% reserved.
% 
% This file is license under the terms of a Creative Commons Attribution 
% 3.0 Unported License, or (at your option) any later version. More details
% at <http://creativecommons.org/licenses/by/3.0/>.
\begin{tabular}{lp{0.8\textwidth}}
    \hline
    Código & Descrição \\ \hline
    \lcode{it} & Texto em itálico. \\
    \lcode{bf} & Texto em negrito. \\
    \lcode{rm} & Texto em romano. \\
    \lcode{sf} & Texto em sans serif. \\
    \lcode{tt} & Texto na tipografia de uma máquina de escrever. \\
    \lcode{sc} & Texto em caixa alta. \\ \hline
\end{tabular}

\end{table}

A seguir é ilustrado as opções apresentadas na Tabela~\ref{tab:text}. \\
\example{codes/hello_fonte@latex_with_vim.tex}

\subsection{Tamanho}
Uma das maneiras de mudar o tamanho da fonte\index{fonte!tamanho} em uma parte
do texto é utilizando um dos ambiente  ou comando de tamanho (a Tabela
\ref{tab:op_tamanho_fonte} apresenta algumas opções disponíveis).
\begin{table}[h!tb]
  \centering
  \caption{Opções disponíveis para o tamanho da fonte, em ordem crescente.}
  \label{tab:op_tamanho_fonte}
  % Filename: font_size@latex_with_vim.tex
% This code is part of LaTeX with Vim.
% 
% Description: LaTeX with Vim is free book about Vim, LaTeX and Git.
% 
% Created: 30.03.12 12:12:36 AM
% Last Change: 30.03.12 12:12:40 AM
% 
% Author: Raniere Gaia Costa da Silva, r.gaia.cs@gmail.com
% Organization:  
% 
% Copyright (c) 2010, 2011, 2012, Raniere Gaia Costa da Silva. All rights 
% reserved.
% 
% This file is license under the terms of a Creative Commons Attribution 
% 3.0 Unported License, or (at your option) any later version. More details
% at <http://creativecommons.org/licenses/by/3.0/>.
\begin{tabular}{lp{0.7\textwidth}}
    \hline
    Código & Descrição \\ \hline
    \lstinline!\tiny! & O menor tamanho possível. \\
    \lstinline!\SMALL! ou \lstinline!\scriptsize! &  \\
    \lstinline!\Small! ou \lstinline!\footnotesize! & Tamanho utilizado em notas de rodapé. \\
    \lstinline!\small! &  \\
    \lstinline!\normalsize! & Tamanho padrão. \\
    \lstinline!\large! & \\
    \lstinline!\Large! & \\
    \lstinline!\LARGE! & \\
    \lstinline!\huge! & \\
    \lstinline!\Huge! & O maior tamanho disponível. \\ \hline
\end{tabular}

\end{table}

Destaca-se que os tamanhos são baseados no tamanho padrão. A seguir um exemplo.
\\
\example{codes/hello_size@latex_with_vim.tex}

\section{Espaçamento}
Nesta seção abordaremos como inserir espaços\index{espaços em branco} ao longo
do texto no LaTeX, mas antes é importante destacar que podemos suprimir espaços
ao utilizar medidas negativas.

\subsection{Espaçamento horizontal}
Para produzir um espaço horizontal utiliza-se o comando
\lstinline!\hspace!\index{comando!hspace@\lstinline+\hspace+} que tem como
parâmetro o tamanho do espaço a ser inserido. Se o comando ocorrer entre duas
linhas ou no início de uma linha o LaTeX não produz o espaço e para este caso
devemos utilizar  \lstinline!\hspace*!.

Para modificar a indentação característica de um novo parágrafo deve-se utilizar
o comando
\begin{code}
  \setlength{\parident}{tam}
\end{code} 
onde \lcode{tam} é o novo tamanho para a indentação dos parágrafos. No caso de
desejar-se suprimir a indentação deve-se utilizar o comando
\lstinline!\noindent!.

O comando \lstinline!\hfill! cria um espaço suficiente para dividir o texto de
modo que o que estiver antes do comando é alinhado a esquerda e o que estiver
depois é alinhado a direita. É permitido utilizar o comando mais de uma vez em
uma linha. O comando é ignorado quando ocorrer entre duas linhas ou no início de
uma linha, neste caso devemos utilizar \lstinline!\hfill*!.

\subsection{Linha horizontal}
Os comandos \lstinline!\dotfill! e \lstinline!\hrulefill! funcionam de maneira
semelhante ao comando \lstinline!\hfill!, mas ao invés de inserir um espaço em
branco é introduzido, respectivamente uma linha pontilhada e uma linha contínua.

\subsection{Espaçamento vertical}
O comando \lstinline!\baselineskip[tam]! estabelece o tamanho do espaçamento
entre linhas para o texto posterior ao comando. Para modificar o tamanho entre
duas linhas específicas pode-se utilizar o comando \lstinline!\\[tam]! inicia
uma nova linha de maneira que \lcode{tam} é o espaçamento entre as linhas.

Para aumentar o espaço entre parágrafos pode-se utilizar um dos comandos
\lstinline!\smallskip!, \lstinline!\medskip! ou \lstinline!\bigskip!, sendo que
o tamanho do espaço está relacionado com o tamanho da fonte padrão do documento.

Os comandos \lstinline!\vspace!\index{comando!vspace@\lstinline+\vspace+} e
\lstinline!\vfill! funcionam, respectivamente, de modo muito semelhante aos
comandos \lstinline!\hspace! e \lstinline!\hfill! só que na vertical.

\subsection{Linha verticais}
O comando \lstinline!\vrule! produz uma linha vertical.

\section{Alinhamento}
Por padrão, o alinhamento\index{alinhamento} ocorre com a margem esquerda e para
alterá-lo pode-se utilizar um dos seguintes ambientes: \envname{center} (para
texto centralizado), \envname{flushleft} (alinhamento a esquerda) e
\envname{flushright} (alinhamento a direita). \\
\example{codes/align@latex_with_vim.tex}

Também é permitido utilizar os comandos: \lstinline!\centering! (para texto
centralizado), \lstinline!\raggedleft! (alinhamento a esquerda) e
\lstinline!\raggedright! (alinhamento a direita).
