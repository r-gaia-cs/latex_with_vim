% Filename: exercises@camecc_lic.tex
% This code is part of 'Cursos CAMECC: Introducao ao LaTeX para o Curso 29 - Licenciatura em Matematica'
% 
% Description: This file correspond to the exercises of the textbook using in the course.
% 
% Created: 07.06.12 11:30:49 AM
% Last Change: 07.06.12 11:30:53 AM
% 
% Authors:
% - Raniere Silva, r.gaia.cs@gmail.com
% 
% Organization: CAMECC - Centro Academico dos Estudantes do IMECC
% 
% Copyright (c) 2012, Raniere Silva. All rights reserved.
% 
% This work is licensed under the Creative Commons Attribution-ShareAlike 3.0 Unported License. To view a copy of this license, visit http://creativecommons.org/licenses/by-sa/3.0/ or send a letter to Creative Commons, 444 Castro Street, Suite 900, Mountain View, California, 94041, USA.
%
% This work is distributed in the hope that it will be useful, but WITHOUT ANY WARRANTY; without even the implied warranty of MERCHANTABILITY or FITNESS FOR A PARTICULAR PURPOSE.
%
\chapter{Exercícios}
Nas páginas a seguir encontram-se alguns exemplos a serem reproduzidos para você testar os comandos e ambientes que foram apresentados neste curso. Como ponto de partida recomendamos utilizar o código abaixo.
\fcode{try/try_minimal.tex}

A seguir algumas dicas referentes aos exemplos presentes nas próximas páginas que não foram cobertos neste curso.
\begin{enumerate}
    \item \label{try11} Utilize os comandos \lstinline!\title!, \lstinline!\author!, \lstinline!\date! e \lstinline!\maketitle! para o título.
    \item \label{try21} Utilize o comando \lstinline!\section! para fazer a divisão do texto e para as referências leia um pouco sobre o BibTeX em \url{http://en.wikibooks.org/wiki/LaTeX/Bibliography_Management}.
    \item \label{try31} Ver o item~\ref{try21} e para a definição das funções seno, cosseno e tangente utilize o ambiente \envname{description}.
\end{enumerate}

\begin{table}[htb]
    \centering
    \begin{tabular}{c}
        Exercício \ref{try11} \\
        \fbox{\includegraphics[height=0.9\textheight]{try/try11.pdf}}
    \end{tabular}
\end{table}
\begin{table}[htb]
    \centering
    \begin{tabular}{c}
        Exercício \ref{try21} \\
        \fbox{\includegraphics[height=0.9\textheight]{try/try21.pdf}}
    \end{tabular}
\end{table}
\begin{table}[htb]
    \centering
    \begin{tabular}{c}
        Exercício \ref{try31}, folha 01 de 02 \\
        \fbox{\includegraphics[page=1, height=0.9\textheight]{try/try31.pdf}}
    \end{tabular}
\end{table}
\begin{table}[htb]
    \centering
    \begin{tabular}{c}
        Exercício \ref{try31}, folha 02 de 02 \\
        \fbox{\includegraphics[page=2, height=0.9\textheight]{try/try31.pdf}}
    \end{tabular}
\end{table}
