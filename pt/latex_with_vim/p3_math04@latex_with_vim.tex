% Filename: p3_math04@latex_with_vim.tex
% This code is part of LaTeX with Vim.
% 
% Description: LaTeX with Vim is free book about Vim, LaTeX and Git.
% 
% Created: 29.03.12 11:44:03 PM
% Last Change: 29.03.12 11:44:23 PM
% 
% Author: Raniere Gaia Costa da Silva, r.gaia.cs@gmail.com
% Organization:  
% 
% Copyright (c) 2010, 2011, 2012, Raniere Gaia Costa da Silva. All rights 
% reserved.
% 
% This file is license under the terms of a Creative Commons Attribution 
% 3.0 Unported License, or (at your option) any later version. More details
% at <http://creativecommons.org/licenses/by/3.0/>.
\chapter{Expressões matemáticas: Parte 4} \label{sch:latex:math4}
\section{Espaço}
No modo matemático do LaTeX, o espaço entre os caracteres é definido de acordo com o tipo a que cada caractere pertence pela própria classificação do LaTeX.

Deve-se tomar cuidado com caracteres $+$, $-$ e $|$ por estes pertencerem a mais de uma classificação no LaTeX.

\subsection{$+$ e $-$}
$+$ e $-$ são operadores quando encontram-se entre símbolos ou um grupo vazio. Nos demais casos, são utilizados como um sinal \\
\begin{minipage}[t]{0.47\linewidth} \vspace{-8pt}
    \begin{latexcode}
        \begin{align*}
            a &{}+{} 1 &= 2 \\
            a &+ 1 &= 2
        \end{align*}
    \end{latexcode}
\end{minipage} \hfill
\begin{minipage}[t]{0.47\linewidth}
    \vspace{0pt}
    \begin{align*}
        a &{}+{} 1 &= 2 \\
        a &+ 1 &= 2
    \end{align*}
\end{minipage}

\subsection{$|$}
Primeiro apresentaremos um exemplo e depois faremos as explicações. \\
\begin{minipage}[t]{0.47\linewidth} \vspace{-8pt}
    \begin{latexcode}
        \begin{align}
            a|b \label{E:bar} \\
            a \mid b \label{E:mid} \\
            \left| a \right| \label{E:leftbar}
        \end{align}
    \end{latexcode}
\end{minipage} \hfill
\begin{minipage}[t]{0.47\linewidth}
    \vspace{0pt}
    \begin{align}
        a|b \label{E:bar} \\
        a \mid b \label{E:mid} \\
        \left| a \right| \label{E:leftbar}
    \end{align}
\end{minipage}

Na equação \ref{E:bar}, observamos o símbolo matemático ``tal que'', em \ref{E:mid} o operador binário ``divisível'' e em \ref{E:leftbar} delimitadores.

\section{Fonte}
No modo matemático, o LaTeX classifica os caracteres em alfabeto matemático e símbolos matemáticos. Baseado nessa classificação escolhe uma fonte a ser usada.

\subsection{Alfabeto matemático}
Para alterar a fonte de caracteres do alfabeto matemático utiliza-se o comando \textsf{\textbackslash mathXX} sendo que \textsf{XX} corresponde ao código da fonte a ser utilizada. A Tabela \ref{tab:op_amfonte} apresenta alguns das opções disponíveis.
\begin{table}[h!tb]
    \centering
    \caption{Opções disponíveis para \textsf{XX} da fonte para o alfabeto matemático.}
    \label{tab:op_amfonte}
    \begin{tabular}{lp{0.8\textwidth}}
        \hline
        Código & Descrição \\ \hline
        \textsf{it} & Texto em itálico. \\
        \textsf{bf} & Texto em negrito. \\
        \textsf{rm} & Texto em romano. \\
        \textsf{sf} & Texto em sans serif. \\
        \textsf{tt} & Texto na tipografia de uma máquina de escrever.
    \end{tabular}
\end{table}

A seguir é ilustrado as opções apresentadas na Tabela \ref{tab:op_amfonte}. \\
\begin{minipage}[t]{0.47\linewidth} \vspace{-8pt}
    \begin{latexcode}
        Normal: $a$. \\
        Italico: $\mathit{a}$. \\
        Negrito: $\mathbf{a}$. \\
        Romano: $\mathrm{a}$. \\
        Sans serif: $\mathsf{a}$. \\
        Maquina de escrever: $\mathtt{a}$.
    \end{latexcode}
\end{minipage} \hfill
\begin{minipage}[t]{0.47\linewidth}
    \vspace{0pt}
    Normal: $a$. \\
    Italico: $\mathit{a}$. \\
    Negrito: $\mathbf{a}$. \\
    Romano: $\mathrm{a}$. \\
    Sans serif: $\mathsf{a}$. \\
    Maquina de escrever: $\mathtt{a}$.
\end{minipage}

\subsection{Símbolos matemáticos}
Para símbolos matemáticos apenas é possível apresentá-los em negrito e, para isso, utiliza-se o comando \textsf{\textbackslash boldsymbol}. \\
\begin{minipage}[t]{0.47\linewidth} \vspace{-8pt}
    \begin{latexcode}
        Normal: $\alpha$. \\
        Negrito: $\boldsymbol{\alpha}$.
    \end{latexcode}
\end{minipage} \hfill
\begin{minipage}[t]{0.47\linewidth}
    \vspace{0pt}
    Normal: $\alpha$. \\
    Negrito: $\boldsymbol{\alpha}$.
\end{minipage}

No LaTeX também existe quatro alfabetos que são interpretados como símbolos. Um deles é o alfabeto grego, apresentado no capítulo anterior e os outros três são acessados com o comando \textsf{\textbackslash mathXX}, sendo que \textsf{XX} corresponde ao código da fonte a ser utilizada. A Tabela \ref{tab:op_asfonte} apresenta as opções disponíveis.
\begin{table}[h!tb]
    \centering
    \caption{Opções disponíveis para \textsf{XX} da fonte para o alfabeto matemático interpretado como símbolo.}
    \label{tab:op_asfonte}
    \begin{tabular}{lp{0.8\textwidth}}
        \hline
        Código & Descrição \\ \hline
        \textsf{cal} & Texto em caligráfico, apenas para caixa alta. \\
        \textsf{frak} & Texto em Euler Fraktur. \\
        \textsf{bb} & Texto em blackboard bold, apenas para caixa alta.
    \end{tabular}
\end{table}

A seguir é ilustrado as opções apresentadas na Tabela \ref{tab:op_asfonte}. \\
\begin{minipage}[t]{0.47\linewidth}\vspace{-8pt}
    \begin{latexcode}
        Normal: $R$. \\
        Caligrafico: $\mathcal{R}$. \\
        Euler Fraktur: $\mathfrak{R}$. \\
        Blackboard bold: $\mathbb{R}$.
    \end{latexcode}
\end{minipage} \hfill
\begin{minipage}[t]{0.47\linewidth}
    \vspace{0pt}
    Normal: $A$. \\
    Caligrafico: $\mathcal{A}$. \\
    Euler Fraktur: $\mathfrak{A}$. \\
    Blackboard bold: $\mathbb{A}$.
\end{minipage}

Destaca-se que a fonte blackboard bold é normalmente utilizada para representar os conjuntos dos números naturais ($\mathbb{N}$), inteiros ($\mathbb{Z}$), reais ($\mathbb{R}$) e complexos ($\mathbb{C}$).

\section{Tamanho}
Em expressões que envolvem frações, índices e expoentes o LaTeX utiliza um tamanho para cada parte da equação. Existem quatro comandos que alteram o tamanho das partes da equação e são apresentados na Tabela \ref{tab:op_mtamanho}
\begin{table}[h!tb]
    \centering
    \caption{Opções disponíveis para o tamanho de partes da equação.}
    \label{tab:op_mtamanho}
    \begin{tabular}{lp{0.8\textwidth}}
        \hline
        Comando & Descrição \\ \hline
        \textbackslash\textsf{displaystyle} & Tamanho padrão para fórmulas displayed. \\
        \textbackslash\textsf{textstyle} & Tamanho padrão para fórmulas inline. \\
        \textbackslash\textsf{scriptstyle} & Tamanho padrão para índices e expoentes. \\
        \textbackslash\textsf{scriptscriptstyle} & Tamanho padrão para índices e expoentes referentes a índices ou expoentes.
    \end{tabular}
\end{table}

Deve-se tomar cuidado ao modificar o tamanho de partes da equação principalmente para fórmulas inline.\\
\begin{minipage}[t]{0.47\linewidth} \vspace{-8pt}
    \begin{latexcode}
        Normal: $\frac{1}{2^2+1}$. \\
        Displaystyle: $\displaystyle \frac{1}{2^2+1}$. \\
        Textstyle: $\textstyle \frac{1}{2^2+1}$. \\
        Scriptstyle: $\scriptstyle \frac{1}{2^2+1}$.
    \end{latexcode}
\end{minipage} \hfill
\begin{minipage}[t]{0.47\linewidth}
    \vspace{0pt}
    Normal: $\frac{1}{2^2+1}$. \\
    Displaystyle: $\displaystyle \frac{1}{2^2+1}$. \\
    Textstyle: $\textstyle \frac{1}{2^2+1}$. \\
    Scriptstyle: $\scriptstyle \frac{1}{2^2+1}$.
\end{minipage}
