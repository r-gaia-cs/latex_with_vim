% Filename: p3_tables@latex_with_vim.tex
% This code is part of LaTeX with Vim.
% 
% Description: LaTeX with Vim is free book about Vim, LaTeX and Git.
% 
% Created: 29.03.12 11:45:34 PM
% Last Change: 29.03.12 11:45:49 PM
% 
% Author: Raniere Gaia Costa da Silva, r.gaia.cs@gmail.com
% Organization:  
% 
% Copyright (c) 2010, 2011, 2012, Raniere Gaia Costa da Silva. All rights 
% reserved.
% 
% This file is license under the terms of a Creative Commons Attribution 
% 3.0 Unported License, or (at your option) any later version. More details
% at <http://creativecommons.org/licenses/by/3.0/>.
\chapter{Tabelas}
Neste capítulos vamos iremos trabalhar com tabelas.

\section{\textsf{tabular}}
O ambiente \textsf{tabular} é utilizado para a construção de tabelas no LaTeX e sua síntaxe é
\begin{lstlisting}[language=TeX]
\begin{tabular}[colunas]
    informacao
\end{tabular}
\end{lstlisting}
onde \textsf{colunas} é uma sequência de caracteres, onde cara caractere corresponde a uma coluna e o respectivo alinhamento que são apresentados na Tabela \ref{tab:par_colunas}, e \textsf{informacao} é o conteudo de cada célula da tabela.
\begin{table}[h!tb]
    \centering
    \caption{Opções disponíveis para \textsf{colunas}.}
    \label{tab:par_colunas}
    \begin{tabular}{lp{0.8\textwidth}}
        \hline
        Código & Descrição \\ \hline
        \textsf{l} & Alinha com margem esquerda. \\
        \textsf{r} & Alinha com a margem direita. \\
        \textsf{c} & Centralizado. \\
        \textsf{p} & Requer como parâmetro a largura da columa. \\
        \textbar & Imprime uma linha separando as colunas. \\ \hline
    \end{tabular}
\end{table}

Cada célula da tabela deve ser separadas pelo comando \& e a mudança de linha ocorre pelo comando \textbackslash\textbackslash. Para imprimir uma linha horizontal separando duas linhas da tabela deve-se utilizar o comando \textbackslash\textsf{hline}.\\
\begin{minipage}[t]{0.47\linewidth}
    \vspace{-8pt}
    \begin{lstlisting}[language=TeX]
    \vspace{0pt}
    \begin{lstlisting}[language=TeX]
    \begin{tabular}{|c|c|c|c|}
        \hline Corrente (A) & Tensao (V) \\ 
        \hline 0,0260 & 14,8 \\
        \hline 0,0246 & 14,0 \\
        \hline 0,0240 & 13,0 \\
        \hline 0,0214 & 12,0 \\
        \hline 
    \end{tabular}
    \end{lstlisting}
\end{minipage} \hfill
\begin{minipage}[t]{0.47\linewidth}
    \vspace{0pt}
    \begin{tabular}{|c|c|c|c|}
        \hline Corrente (A) & Tensao (V) \\ 
        \hline 0,0260 & 14,8 \\
        \hline 0,0246 & 14,0 \\
        \hline 0,0240 & 13,0 \\
        \hline 0,0214 & 12,0 \\
        \hline 
    \end{tabular} 
\end{minipage} 

Outros comandos também são importantes para a construção mas não trataremos deles aqui.

\section{\textsf{table}}
O ambiente \textsf{table} possibilita a inclusão de uma legenda para a tabela e trabalha a mesma como um objeto flutuante. A síntaxe deste ambiente, muito semelhante com a do ambiente \textsf{figure}, é
\begin{lstlisting}[language=TeX]
\begin{table}[place]
    tabela
    \caption{legend}
    \label{P:tebela}
\end{table}
\end{lstlisting}
onde \textsf{place} é o parâmetro que indica onde a tabela deve ser preferencialmente inserida (as opções disponíveis são apresentadas na Tabela \ref{tab:par_place_tab} e a opção padrão é \textsf{tbp}), \textsf{tabela} corresponde ao código da tabela a ser inserida, \textbackslash\textsf{caption} é o comando correspondente a legenda e \textsf{legend} é o texto a ser apresentado como legenda, \textbackslash\textsf{label} é o comando para referência cruzada como já apresentado.
\begin{table}[!htb]
    \centering
    \caption{Opções disponíveis para \textsf{place}.}
    \label{tab:par_place_tab}
    \begin{tabular}{lp{0.8\textwidth}}
        \hline
        Código & Descrição \\ \hline
        \textsf{h} & Na posição onde o código se encontra. \\
        \textsf{t} & No topo de uma página. \\
        \textsf{b} & No fim de uma página. \\
        \textsf{p} & Em uma página separada. \\
        \textsf{!} & Modifica algumas configurações a respeito de boa posição para objeto flutuante. \\ \hline
    \end{tabular}
\end{table}

O código a seguir
\begin{lstlisting}[language=TeX]
\begin{table}[h!tb] \label{T:tab_exemp} \centering
    \caption{Relacao entre corrente e tensao para determinado circuito.}
    \begin{tabular}{|c|c|c|c|}
        \hline Corrente (A) & Tensao (V) \\ 
        \hline 0,0260 & 14,8 \\
        \hline 0,0246 & 14,0 \\
        \hline 0,0240 & 13,0 \\
        \hline 0,0214 & 12,0 \\
        \hline 
    \end{tabular}
\end{table}
\end{lstlisting}
produz a tabela abaixo.
\begin{table}[h!tb] \label{T:tab_exemp} \centering
    \caption{Relacao entre corrente e tensao para determinado circuito.}
    \begin{tabular}{|c|c|c|c|}
        \hline Corrente (A) & Tensao (V) \\ 
        \hline 0,0260 & 14,8 \\
        \hline 0,0246 & 14,0 \\
        \hline 0,0240 & 13,0 \\
        \hline 0,0214 & 12,0 \\
        \hline 
    \end{tabular}
\end{table}

Uma dica útil é que o comando \textbackslash\textsf{clearpage} força as tabelas pendentes a serem inseridas.

\section{\textsf{Openoffice}}

Muitas vezes temos uma tabela no \textsf{Calc}\footnote{O Calc é um dos aplicativos do pacote \textsf{Openoffice} e corresponde ao popular \textsf{Excel} do pacote \textsf{Microsoft Office}.} e desejamos transportá-la para o LaTeX. Para essa tarefa a macro \textsf{Calc2LaTeX}, disponível gratuitamente em \url{http://extensions.services.openoffice.org/en/project/Calc2LaTeX}, é bastante eficiente.
