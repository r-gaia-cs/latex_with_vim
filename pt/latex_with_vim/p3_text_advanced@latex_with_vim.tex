% Filename: p3_text_advanced@latex_with_vim.tex
% This code is part of LaTeX with Vim.
% 
% Description: LaTeX with Vim is free book about Vim, LaTeX and Git.
% 
% Created: 29.03.12 11:46:50 PM
% Last Change: 29.03.12 11:47:06 PM
% 
% Author: Raniere Gaia Costa da Silva, r.gaia.cs@gmail.com
% Organization:  
% 
% Copyright (c) 2010, 2011, 2012, Raniere Gaia Costa da Silva. All rights 
% reserved.
% 
% This file is license under the terms of a Creative Commons Attribution 
% 3.0 Unported License, or (at your option) any later version. More details
% at <http://creativecommons.org/licenses/by/3.0/>.
\chapter{Trabalhando um texto: Parte 2} \label{C:text2}
Neste capítulo continuaremos apresentado algumas ferramentas do LaTeX disponíveis para organizar um texto, agora voltadas para textos matemáticos, e apresentaremos o ambeinte \textsf{minipage}.

\section{Nota de rodapé}

Para produzir notas de rodapé deve-se utilizar o comando \textbackslash\textsf{footnote} que deve ocorrer imediatamente depois da palavra ou texto a que se refere a nota de rodapé e como parâmetro do comando o texto a ser inserido na nota de rodapé.

\section{Referência cruzada} \label{sse:cross_reference}

Existem dois tipos de referência cruzada, a primeira para alguma parte do documento e a segunda para um outro documento. Nesta seção abordaremos o primeiro tipo e o segundo será trabalhado no Capítulo~\ref{sch:latex::bibtex}.

Para alguns comandos e ambientes o LaTeX atribui um número que pode ser vinculado a um nome pelo comando \textbackslash\textsf{label} e referenciado pelo comando \textbackslash\textsf{ref} ou \textbackslash\textsf{pageref}, para a página onde encontra-se o item referenciado.

O argumento do comando \textbackslash\textsf{label} é uma sequencia de caracteres\footnote{Recomenda-se escolher uma sequencia ``amigável''}, case sensitive, que será utilizada como argumento do comando \textbackslash\textsf{ref} ao efetuar a referência.

Ao utilizar os comandos \textbackslash\textsf{ref} ou \textbackslash\textsf{pageref} é aconselhavel precedé-los por um \verb+~+.

\section{Citações}

No LaTeX encontramos dois ambientes dedicados a citações. O primeiro deles é o \textsf{quote} próprío para citações de uma única linha e o segundo é o \textsf{quotation} adequado para citações de vários parágrafos.

\section{Poemas}

O ambiente \textsf{verse} é apropriado para poemas uma vez que nestes a separação entre linhas é essencial. Os versos são divididos pelo comando \textbackslash\textbackslash \ e as estrofes com linhas em branco.

\section{Listas}

Para a construção de listas podemos utilizar um dos quatro ambientes: \textsf{itemize}, \textsf{enumerate}, \textsf{description} ou \textsf{list}. E para a criação de sublistas basta adicionar um dos ambientes dentro de um já existente.

Cada item de uma lista é identificado, no LaTeX, pelo comando \textbackslash\textsf{item} que deve preceder o texto.

\subsection{\textsf{itemize}}
O ambiente \textsf{itemize} utiliza um símbolo para indicar cada item da lista. \\
\begin{minipage}[t]{0.47\linewidth} \vspace{-8pt}
    \begin{lstlisting}[language=TeX]
    \begin{itemize}
        \item Primeiro;
            \begin{itemize}
                \item Subitem;
            \end{itemize}
        \item Segundo.
    \end{itemize}
    \end{lstlisting}
\end{minipage} \hfill
\begin{minipage}[t]{0.47\linewidth} \vspace{0pt}
    \begin{itemize}
        \item Primeiro;
            \begin{itemize}
                \item Subitem;
            \end{itemize}
        \item Segundo.
    \end{itemize}
\end{minipage}

\subsection{\textsf{enumerate}}
O ambiente \textsf{enumerate} numera cada um dos itens da lista. \\
\begin{minipage}[t]{0.47\linewidth} \vspace{-8pt}
    \begin{lstlisting}[language=TeX]
    \begin{enumerate}
        \item Primeiro;
            \begin{enumerate}
                \item Subitem;
            \end{enumerate}
        \item Segundo.
    \end{enumerate}
    \end{lstlisting}
\end{minipage} \hfill
\begin{minipage}[t]{0.47\linewidth} \vspace{0pt}
    \begin{enumerate}
        \item Primeiro;
            \begin{enumerate}
                \item Subitem;
            \end{enumerate}
        \item Segundo.
    \end{enumerate}
\end{minipage}

\subsection{\textsf{description}}
O ambiente \textsf{desciption} é adequado para a descrição de item de uma lista uma vez que cada item da lista é identificado por uma sequencia de caracteres negritado. \\
\begin{minipage}[t]{0.47\linewidth} \vspace{-8pt}
    \begin{lstlisting}[language=TeX]
    \begin{description}
        \item[Um] Primeiro;
            \begin{description}
                \item[Um.Um] Subitem;
            \end{description}
        \item[Dois] Segundo.
    \end{description}
    \end{lstlisting}
\end{minipage} \hfill
\begin{minipage}[t]{0.47\linewidth} \vspace{0pt}
    \begin{description}
        \item[Um] Primeiro;
            \begin{description}
                \item[Um.Um] Subitem;
            \end{description}
        \item[Dois] Segundo.
    \end{description}
\end{minipage}

\subsection{\textsf{list}}
O ambiente \textsf{list} é normalmente utilizado em macros, mas pode ser utilizado em documentos também, e apropriado para a criação de uma lista na qual precisa-se de um espaçamento específico. Maiores informações podem ser encontradas em \url{http://www.troubleshooters.com/linux/lyx/ownlists.htm}.
