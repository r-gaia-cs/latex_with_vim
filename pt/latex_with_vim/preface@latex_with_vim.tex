% Filename: preface@latex_with_vim.tex
% This code is part of LaTeX with Vim.
% 
% Description: LaTeX with Vim is free book about Vim, LaTeX and Git.
% 
% Created: 29.03.12 11:52:59 PM
% Last Change: 29.03.12 11:53:32 PM
% 
% Author: Raniere Gaia Costa da Silva, r.gaia.cs@gmail.com
% Organization:  
% 
% Copyright (c) 2010, 2011, 2012, Raniere Gaia Costa da Silva. All rights 
% reserved.
% 
% This file is license under the terms of a Creative Commons Attribution 
% 3.0 Unported License, or (at your option) any later version. More details
% at <http://creativecommons.org/licenses/by/3.0/>.
\chapter*{Prefácio}
\addcontentsline{toc}{chapter}{Prefácio}

Meu primeiro contato com o Latex foi no primeiro ano de faculdade. Na época, me foi passado alguns tutoriais sobre Latex os quais considerei muito aquém do desejado.

Nos dois primeiros anos de faculdade, fui guardando algumas dicas úteis para trabalhar com o Latex. No segundo ano de faculdade tomei conhecimento do editor de textos Vim e logo em seguida do LaTeX-Suite, desenvolvido para o mesmo, que me facilitou muito trabalhar com o LaTeX. Aqui apresento algumas dicas do LaTeX e Vim que fui guardando nestes breves dois anos.

%No segundo ano de faculdade, tomei conhecimento do editor de texto VIM, que possui um conjunto de macros, dos quais gostei muito e por isso dou algumas dicas de como utilizá-lo também.

Na primeira parte aboraderemos o LaTeX e na segunda o Vim juntamente com o LaTeX-Suite.

Encontrar uma boa maneira de apresentar os comandos, ambientes e dicas do LaTeX foi uma tarefa difícil. Optou-se por começar preparando o sistema operacional para trabalhar com o LaTeX e em seguida como trabalhar com vários arquivos.

No capítulo \ref{sch:latex:preamble} abordamos o preâmbulo e no capítulo seguinte conhecimentos básicos para escrever uma carta. Outras ferramentas do LaTeX são apresentadas nos capítulos  \ref{css:latex:text_basic} e \ref{css:latex:text_basic} enquanto que no capítulo \ref{sch:latex:publi} abordamos comandos para a produção de teses e livros.

Informações sobre o modo matemático encontram-se nos capítulos \ref{sch:latex:math1}, \ref{sch:latex:math2}, \ref{sch:latex:math3} e \ref{sch:latex:math4}. No primeiro encontram-se os comandos básicos enquanto que no segundo apresentamos tabelas outros comandos. Os dois outros capítulos apresentam dicas para que as expressões matemáticas sejam apresentadas de maneira esteticamente melhor.

Depois de abordarmos o modo matemático temos os capítulos referentes a tabelas, figuras, códigos, algorítmos e bibliografia.

Já na segunda parte, abordamos nos três primeiros capítulos o Vim e nas duas últimas o LaTeX-Suite.

\subsection*{Agradecimentos}
Meus agradecimentos a todos que me motivaram a este trabalho e com ele contribuíram. Um muito obrigado para Marina Lima Morais pelo apoio na revisão do trabalho.
