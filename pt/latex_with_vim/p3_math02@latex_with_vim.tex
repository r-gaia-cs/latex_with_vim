% Filename: p3_math02@latex_with_vim.tex
% This code is part of LaTeX with Vim.
% 
% Description: LaTeX with Vim is free book about Vim, LaTeX and Git.
% 
% Created: 29.03.12 11:41:27 PM
% Last Change: 29.03.12 11:42:24 PM
% 
% Author: Raniere Gaia Costa da Silva, r.gaia.cs@gmail.com
% Organization:  
% 
% Copyright (c) 2010, 2011, 2012, Raniere Gaia Costa da Silva. All rights 
% reserved.
% 
% This file is license under the terms of a Creative Commons Attribution 
% 3.0 Unported License, or (at your option) any later version. More details
% at <http://creativecommons.org/licenses/by/3.0/>.
\chapter{Expressões matemáticas: Parte 2} \label{sch:latex:math2}
Neste capítulo continuaremos trabalhando com fórmulas matemáticas e nosso objetivo é apresentar alguns comandos úteis.

\section{Relações binárias}
\begin{table}[h!tbp]
    \caption{Relações binárias} \centering
    \label{tab:math_binary_relations}
    % Filename: math_binary_relations@latex_with_vim.tex
% This code is part of LaTeX with Vim.
% 
% Description: LaTeX with Vim is free book about Vim, LaTeX and Git.
% 
% Created: 30.03.12 12:15:58 AM
% Last Change: 30.03.12 12:16:09 AM
% 
% Author: Raniere Gaia Costa da Silva, r.gaia.cs@gmail.com
% Organization:  
% 
% Copyright (c) 2010, 2011, 2012, Raniere Gaia Costa da Silva. All rights 
% reserved.
% 
% This file is license under the terms of a Creative Commons Attribution 
% 3.0 Unported License, or (at your option) any later version. More details
% at <http://creativecommons.org/licenses/by/3.0/>.
\begin{tabular}{cc|cc|cc}
    \hline
    Com. & Res. & Com. & Res. & Com. & Res. \\ \hline
    \lstinline!<! & $<$ & \lstinline!\nless! & $\nless$ & \lstinline!>! & $>$ \\
    \lstinline!\ngtr! & $\ngtr$ & \lstinline!\ll! & $\ll$ & \lstinline!\lll! & $\lll$ \\
    \lstinline!\gg! & $\gg$ & \lstinline!\ggg! & $\ggg$ & \lstinline!=! & $=$ \\
    \lstinline!\neq! & $\neq$ & \lstinline!:! & $:$ & \lstinline!\doteq! & $\doteq$ \\
    \lstinline!\sim! & $\sim$ & \lstinline!\nsim! & $\nsim$ & \lstinline!\cong! & $\cong$ \\
    \lstinline!\ncong! & $\ncong$ & \lstinline!\simeq! & $\simeq$ & \lstinline!\approx! & $\approx$ \\
    \lstinline!\equiv! & $\equiv$ & \lstinline!\leq! ou \lstinline!\le! & $\leq$ & \lstinline!\nleq! & $\nleq$ \\
    \lstinline!\geq! ou \lstinline!\ge! & $\geq$ & \lstinline!\ngeq! & $\ngeq$ & \lstinline!\leqslant! & $\leqslant$ \\
    \lstinline!\nleqslant! & $\nleqslant$ & \lstinline!\geqslant! & $\geqslant$ & \lstinline!\ngeqslant! & $\ngeqslant$ \\
    \lstinline!\eqslantless! & $\eqslantless$ & \lstinline!\eqslantgtr! & $\eqslantgtr$ & \lstinline!\leqq! & $\leqq$ \\
    \lstinline!\nleqq! & $\nleqq$ & \lstinline!\geqq! & $\geqq$ & \lstinline!\ngeqq! & $\ngeqq$ \\
    \lstinline!\lesssim! & $\lesssim$ & \lstinline!\lessapprox! & $\lessapprox$ & \lstinline!\gtrsim! & $\gtrsim$ \\
    \lstinline!\gtrapprox! & $\gtrapprox$ & \lstinline!\prec! & $\prec$ & \lstinline!\nprec! & $\nprec$ \\
    \lstinline!\succ! & $\succ$ & \lstinline!\nsucc! & $\nsucc$ & \lstinline!\preceq! & $\preceq$ \\
    \lstinline!\npreceq! & $\npreceq$  & \lstinline!\succeq! & $\succeq$ & \lstinline!\nsucceq! & $\nsucceq$ \\
    \lstinline!\in! & $\in$ & \lstinline!\notin! & $\notin$ & \lstinline!\owns! & $\owns$ \\
    \lstinline!\subset! & $\subset$ & \lstinline!\supset! & $\supset$ & \lstinline!\subseteq! & $\subseteq$ \\
    \lstinline!\nsubseteq! & $\nsubseteq$ & \lstinline!\supseteq! & $\supseteq$ & \lstinline!\nsupseteq! & $\nsupseteq$ \\
    \lstinline!\subseteqq! & $\subseteqq$ & \lstinline!\nsubseteqq! & $\nsubseteqq$ & \lstinline!\supseteqq! & $\supseteqq$ \\
    \lstinline!\nsupseteqq! & $\nsupseteqq$ & \lstinline!\sqsubset! & $\sqsubset$ & \lstinline!\sqsubseteq! & $\sqsubseteq$ \\
    \lstinline!\sqsupset! & $\sqsupset$ & \lstinline!\sqsupseteq! & $\sqsupseteq$ & \lstinline!\smile! & $\smile$ \\
    \lstinline!\smallsmile! & $\smallsmile$ & \lstinline!\frown! & $\frown$ & \lstinline!\smallfrown! & $\smallfrown$ \\
    \lstinline!\perp! & $\perp$ & \lstinline!\models! & $\models$ & \lstinline!\mid! & $\mid$ \\
    \lstinline!\nmid! & $\nmid$ & \lstinline!\parallel! & $\parallel$ & \lstinline!\nparallel! & $\nparallel$ \\
    \lstinline!\shortmid! & $\shortmid$ & \lstinline!\nshortmid! & $\nshortmid$ & \lstinline!\shortparallel! & $\shortparallel$ \\
    \lstinline!\nshortparallel! & $\nshortparallel$ & \lstinline!\vdash! & $\vdash$ & \lstinline!\nvdash! & $\nvdash$ \\
    \lstinline!\dashv! & $\dashv$ & \lstinline!\vDash! & $\vDash$ & \lstinline!\nvDash! & $\nvDash$ \\
    \lstinline!\Vdash! & $\Vdash $ & \lstinline!\nVdash! & $\nVdash$ & \lstinline!\propto! & $\propto$ \\
    \lstinline!\asymp! & $\asymp$ & \lstinline!\bowtie! & $\bowtie$ & \lstinline!\Join! & $\Join$ \\
    \lstinline!\vartriangleleft! & $\vartriangleleft$ & \lstinline!\ntriangleleft! & $\ntriangleleft$ & \lstinline!\vartriangleright! & $\vartriangleright$ \\
    \lstinline!\ntriangleright! & $\ntriangleright$ & \lstinline!\trianglelefteq! & $\trianglelefteq$ & \lstinline!\ntrianglelefteq! & $\ntrianglelefteq$ \\
    \lstinline!\trianglerighteq! & $\trianglerighteq$ & \lstinline!\ntrianglerighteq! & $\ntrianglerighteq$ &  \lstinline!\blacktriangleleft! & $\blacktriangleleft$ \\
    \lstinline!\blacktriangleright! & $\blacktriangleright$ & \lstinline!\between! & $\between$ & \lstinline!\pitchfork! & $\pitchfork$ \\
    \lstinline!\therefore! & $\therefore$ & \lstinline!\because! & $\because$ \\ \hline
\end{tabular}
\begin{flushleft}
    Enquanto que \lstinline!|! \'{e} um limitador, \lstinline!\mid! \'{e} um operador que corresponde a express\~{a}o ``tal que''.
\end{flushleft}

\end{table}

\section{Operadores binários}
\begin{table}[h!tb]
    \centering
    \caption{Operadores binários}
    \label{tab:math_binary_operations}
    % Filename: math_binary_operations@latex_with_vim.tex
% This code is part of LaTeX with Vim.
% 
% Description: LaTeX with Vim is free book about Vim, LaTeX and Git.
% 
% Created: 30.03.12 12:15:37 AM
% Last Change: 30.03.12 12:15:42 AM
% 
% Author: Raniere Gaia Costa da Silva, r.gaia.cs@gmail.com
% Organization:  
% 
% Copyright (c) 2010, 2011, 2012, Raniere Gaia Costa da Silva. All rights 
% reserved.
% 
% This file is license under the terms of a Creative Commons Attribution 
% 3.0 Unported License, or (at your option) any later version. More details
% at <http://creativecommons.org/licenses/by/3.0/>.
\begin{tabular}{cc|cc|cc}
    \hline
    Com. & Res. & Com. & Res. & Com. & Res. \\ \hline
    \lstinline!+! & $+$ & \lstinline!-! & $-$ & \lstinline!\pm! & $\pm$ \\
    \lstinline!\mp! & $\mp$ & \lstinline!\times! & $\times$ & \lstinline!\cdot! & $\cdot$ \\
    \lstinline!\div! & $\div$ & \lstinline!\And! & $\And$ & \lstinline!\setminus! & $\setminus$ \\
    \lstinline!\smallsetminus! & $\smallsetminus$ & \lstinline!\dagger! & $\dagger$ & \lstinline!\ddagger! & $\ddagger$ \\
    \lstinline!\ast! & $\ast$ & \lstinline!\star! & $\star$ & \lstinline!\wedge! & $\wedge$ \\
    \lstinline!\vee! & $\vee$ & \lstinline!\cap! & $\cap$ & \lstinline!\cup! & $\cup$ \\
    \lstinline!\sqcap! & $\sqcap$ & \lstinline!\sqcup! & $\sqcup$ & \lstinline!\oplus! & $\oplus$ \\
    \lstinline!\ominus! & $\ominus$ & \lstinline!\otimes! & $\otimes$ & \lstinline!\oslash! & $\oslash$ \\
    \lstinline!\odot! & $\odot$ & \lstinline!\bigcirc! & $\bigcirc$ & \lstinline!\circ! & $\circ$ \\
    \lstinline!\bullet! & $\bullet$ & \lstinline!\bigtriangleup! & $\bigtriangleup$ & \lstinline!\bigtriangledown! & $\bigtriangledown$ \\
    \lstinline!\triangleleft! & $\triangleleft$ & \lstinline!\triangleright! & $\triangleright$ &\lstinline!\diamond! & $\diamond$ \\
    \lstinline!\wr! & $\wr$ & \lstinline!\amalg! & $\amalg$ \\ \hline
\end{tabular}

\end{table}

\section{Outros operadores}

No LaTeX encontramos três categorias de operadores: \textit{puros}, não precisam de parâmetro, \textit{com intervalos}, são definidos apenas em um intervalo e para indicá-lo utilizas-se underscore e caret, e \textit{limites}. Nas Tabelas \ref{tab:math_functions1}, \ref{tab:math_functions2} e \ref{tab:math_functions3} apresentamos os operadores disponíveis no LaTeX.

\begin{table}[h!tb]
    \centering
    \caption{Operadores \textit{puros}.}
    \label{tab:math_functions1}
    % Filename: math_functions1@latex_with_vim.tex
% This code is part of LaTeX with Vim.
% 
% Description: LaTeX with Vim is free book about Vim, LaTeX and Git.
% 
% Created: 30.03.12 12:16:46 AM
% Last Change: 30.03.12 12:16:51 AM
% 
% Author: Raniere Gaia Costa da Silva, r.gaia.cs@gmail.com
% Organization:  
% 
% Copyright (c) 2010, 2011, 2012, Raniere Gaia Costa da Silva. All rights 
% reserved.
% 
% This file is license under the terms of a Creative Commons Attribution 
% 3.0 Unported License, or (at your option) any later version. More details
% at <http://creativecommons.org/licenses/by/3.0/>.
\begin{tabular}{cc|cc|cc}
    \hline
    Com. & Res. & Com. & Res. & Com. & Res. \\ \hline
    \lstinline!\log! & $\log$ & \lstinline!\ln! & $\ln$ & \lstinline!\exp! & $\exp$ \\
    \lstinline!\arccos! & $\arccos$ & \lstinline!\arcsin! & $\arcsin$ & \lstinline!\arctan! & $\arctan$ \\
    \lstinline!\cos! & $\cos$ & \lstinline!\sin! & $\sin$ & \lstinline!\tan! & $\tan$ \\
    \lstinline!\csc! & $\csc$ & \lstinline!\sec! & $\sec$ & \lstinline!\cot! & $\cot$ \\
    \lstinline!\cosh! & $\cosh$ & \lstinline!\sinh! & $\sinh$ & \lstinline!\tanh! & $\tanh$ \\
    \lstinline!\lg! & $\lg$ & \lstinline!\arg! & $\arg$ & \lstinline!\hom! & $\hom$ \\
    \lstinline!\dim! & $\dim$ & \lstinline!\ker! & $\ker$ & \lstinline!\det! & $\det$ \\
    \lstinline!\gcd! & $\gcd$ & & & & \\ \hline
\end{tabular}

\end{table}

\begin{table}[h!tb]
    \centering
    \caption{Operadores \textit{com intervalos}.}
    \label{tab:math_functions2}
    % Filename: math_functions2@latex_with_vim.tex
% This code is part of LaTeX with Vim.
% 
% Description: LaTeX with Vim is free book about Vim, LaTeX and Git.
% 
% Created: 30.03.12 12:17:02 AM
% Last Change: 30.03.12 12:17:06 AM
% 
% Author: Raniere Gaia Costa da Silva, r.gaia.cs@gmail.com
% Organization:  
% 
% Copyright (c) 2010, 2011, 2012, Raniere Gaia Costa da Silva. All rights 
% reserved.
% 
% This file is license under the terms of a Creative Commons Attribution 
% 3.0 Unported License, or (at your option) any later version. More details
% at <http://creativecommons.org/licenses/by/3.0/>.
\begin{tabular}{cc|cc|cc|cc}
    \hline
    Comando & Resultado & Comando & Resultado & Comando & Resultado & Comando & Resultado \\ \hline
    \textbackslash\textsf{int} & $\int$ & \textbackslash\textsf{iint} & $\iint$ & \textbackslash\textsf{iiint} & $\iiint$ & \textbackslash\textsf{iiiint} & $\iiiint$ \\
    \textbackslash\textsf{idotsint} & $\idotsint$ & \textbackslash\textsf{oint} & $\oint$ & \textbackslash\textsf{prod} & $\prod$ & \textbackslash\textsf{coprod} & $\coprod$ \\ 
    \textbackslash\textsf{bigcap} & $\bigcap$ & \textbackslash\textsf{bigcup} & $\bigcup$ & \textbackslash\textsf{bigwedge} & $\bigwedge$ & \textbackslash\textsf{bigvee} & $\bigvee$ \\ 
    \textbackslash\textsf{bigsqcup} & $\bigsqcup$ & \textbackslash\textsf{biguplus} & $\biguplus$ & \textbackslash\textsf{bigotimes} & $\bigotimes$ & \textbackslash\textsf{bigoplus} & $\bigoplus$ \\ 
    \textbackslash\textsf{bigodot} & $\bigodot$ & \textbackslash\textsf{sum} & $\sum$ & & & & \\ \hline
\end{tabular}

\end{table}

\begin{table}[!htb]
    \centering
    \caption{\textit{Limites}.}
    \label{tab:math_functions3}
    % Filename: math_functions3@latex_with_vim.tex
% This code is part of LaTeX with Vim.
% 
% Description: LaTeX with Vim is free book about Vim, LaTeX and Git.
% 
% Created: 30.03.12 12:17:16 AM
% Last Change: 30.03.12 12:17:21 AM
% 
% Author: Raniere Gaia Costa da Silva, r.gaia.cs@gmail.com
% Organization:  
% 
% Copyright (c) 2010, 2011, 2012, Raniere Gaia Costa da Silva. All rights 
% reserved.
% 
% This file is license under the terms of a Creative Commons Attribution 
% 3.0 Unported License, or (at your option) any later version. More details
% at <http://creativecommons.org/licenses/by/3.0/>.
\begin{tabular}{cc|cc|cc|cc}
    \hline
    Comando & Resultado & Comando & Resultado & Comando & Resultado \\ \hline
    \textbackslash\textsf{lim} & $\lim$ & \textbackslash\textsf{inf} & $\inf$ & \textbackslash\textsf{sup} & $\sup$ & \textbackslash\textsf{max} & $\max$ \\
    \textbackslash\textsf{injlim} & $\injlim$ & \textbackslash\textsf{liminf} & $\liminf$ & \textbackslash\textsf{limsup} & $limsup$ & \textbackslash\textsf{min} & $\min$ \\
    \textbackslash\textsf{varinjlim} & $\varinjlim$ & \textbackslash\textsf{varliminf} & $\varliminf$ & \textbackslash\textsf{varlimsup} & $varlimsup$ & \textbackslash\textsf{Pr} & $\Pr$ \\
    \textbackslash\textsf{projlim} & $\projlim$ & \textbackslash\textsf{varprojlim} & $\varprojlim$ & \\ \hline
\end{tabular}

\end{table}

Além destes operadores ainda temos o comando \textbackslash\textsf{sqrt} para raízes.

A seguir alguns exemplos. \\
\begin{minipage}[t]{0.47\linewidth} \vspace{-8pt}
    \begin{latexcode}
        $\sin \frac{\pi}{2} = 1$ \\
        $\int_0^\pi \sin x \, dx = 2$ \\
        $\lim_{x \to 0} x = 0$ \\
        $\sqrt{4} = 2$ \\
        $\sqrt[3]{8} = 2$
    \end{latexcode}
\end{minipage} \hfill
\begin{minipage}[t]{0.47\linewidth} \vspace{0pt}
    $\sin \frac{\pi}{2} = 1$ \\
    $\int_0^\pi \sin x \, dx = 2$ \\
    $\lim_{x \to 0} x = 0$ \\
    $\sqrt{4} = 2$ \\
    $\sqrt[3]{8} = 2$
\end{minipage}

O LaTeX também permite definirmos novos operadores \textit{puros} pelo comando
\begin{latexcode}
    \DeclareMathOperator{comando}{resultado}
\end{latexcode}
onde \textsf{comando} corresponde ao comando a ser utilizado para o novo operador e \textsf{resultado} é o texto apresentado como resultado. Este comando é muito utilizado para a tradução de alguns operadores, como por exemplo o seno.

\subsection{Binomial}
Utiliza-se o comando \textbackslash\textsf{binom} para os binomios. \\
\begin{minipage}[t]{0.47\linewidth} \vspace{-8pt}
    \begin{latexcode}
        $\binom{a}{b} + \binom{a+1}{b} = \binom{a+1}{b+1}$
    \end{latexcode}
\end{minipage} \hfill
\begin{minipage}[t]{0.47\linewidth} \vspace{0pt}
    $\binom{a}{b} + \binom{a+1}{b} = \binom{a+1}{b+1}$
\end{minipage}

\subsection{Congruências}
A forma mais comum para congruências corresponde ao uso dos comandos \textbackslash\textsf{equiv} e \textbackslash\textsf{pmod}. \\
\begin{minipage}[t]{0.47\linewidth} \vspace{-8pt}
    \begin{latexcode}
        $a \equiv b \pmod{v}$
    \end{latexcode}
\end{minipage} \hfill
\begin{minipage}[t]{0.47\linewidth} \vspace{0pt}
    $a \equiv b \pmod{v}$
\end{minipage}

\section{Setas}
\begin{table}[h!tb]
    \centering
    \caption{Setas}
    \label{tab:math_arrows}
    % Filename: math_arrows@latex_with_vim.tex
% This code is part of LaTeX with Vim.
% 
% Description: LaTeX with Vim is free book about Vim, LaTeX and Git.
% 
% Created: 30.03.12 12:15:12 AM
% Last Change: 30.03.12 12:15:22 AM
% 
% Author: Raniere Gaia Costa da Silva, r.gaia.cs@gmail.com
% Organization:  
% 
% Copyright (c) 2010, 2011, 2012, Raniere Gaia Costa da Silva. All rights 
% reserved.
% 
% This file is license under the terms of a Creative Commons Attribution 
% 3.0 Unported License, or (at your option) any later version. More details
% at <http://creativecommons.org/licenses/by/3.0/>.
\begin{tabular}{cc|cc|cc}
    \hline
    Com. & Res. & Com. & Res. & Com. & Res. \\ \hline
    \lstinline!\leftarrow! & $\leftarrow$ & \lstinline!\rightarrow! & $\rightarrow$ & \lstinline!\longleftarrow! & $\longleftarrow$ \\
    \lstinline!\longrightarrow! & $\longrightarrow$ & \lstinline!\Leftarrow! & $\Leftarrow$ & \lstinline!\Rightarrow! & $\Rightarrow$ \\
    \lstinline!\Longleftarrow! & $\Longleftarrow$ & \lstinline!\Longrightarrow! & $\Longrightarrow$ & \lstinline!\nleftarrow! & $\nleftarrow$ \\
    \lstinline!\nrightarrow! & $\nrightarrow$ & \lstinline!\nLeftarrow! & $\nLeftarrow$ & \lstinline!\nRightarrow! & $\nRightarrow$ \\
    \lstinline!\leftrightarrow! & $\leftrightarrow$ & \lstinline!\longleftrightarrow! & $\longleftrightarrow$ & \lstinline!\Leftrightarrow! & $\Leftrightarrow$ \\
    \lstinline!\Longleftrightarrow! & $\Longleftrightarrow$ & \lstinline!\nleftrightarrow! & $\nleftrightarrow$ & \lstinline!\nLeftrightarrow! & $\nLeftrightarrow$ \\
    \lstinline!\dashleftarrow! & $\dashleftarrow$ & \lstinline!\dashrightarrow! & $\dashrightarrow$ & \lstinline!\leftrightharpoons! & $\leftrightharpoons$ \\
    \lstinline!\rightleftharpoons! & $\rightleftharpoons$ & \lstinline!\leftrightarrows! & $\leftrightarrows$ & \lstinline!\rightleftarrows! & $\rightleftarrows$ \\
    \lstinline!\mapsto! & $\mapsto$ & \lstinline!\longmapsto! & $\longmapsto$ & \lstinline!\iff! & $\iff$ \\
    \lstinline!\uparrow! & $\uparrow$ & \lstinline!\downarrow! & $\downarrow$ & \lstinline!\Uparrow! & $\Uparrow$ \\
    \lstinline!\Downarrow! & $\Downarrow$ & \lstinline!\updownarrow! & $\updownarrow$ & \lstinline!\Updownarrow! & $\Updownarrow$ \\
    \lstinline!\Lsh! & $\Lsh$ & \lstinline!\Rsh! & $\Rsh$ & \lstinline!\curvearrowleft! & $\curvearrowleft$ \\
    \lstinline!\curvearrowright! & $\curvearrowright$ & \lstinline!\circlearrowleft! & $\circlearrowleft$ & \lstinline!\circlearrowright! & $\circlearrowright$ \\ \hline
\end{tabular}

\end{table}

\section{Outros símbolos}
\begin{table}[!htb]
    \centering
    \caption{Outros símbolos matemáticos}
    \label{tab:math_others}
    % Filename: math_others@latex_with_vim.tex
% This code is part of LaTeX with Vim.
% 
% Description: LaTeX with Vim is free book about Vim, LaTeX and Git.
% 
% Created: 30.03.12 12:18:26 AM
% Last Change: 30.03.12 12:18:30 AM
% 
% Author: Raniere Gaia Costa da Silva, r.gaia.cs@gmail.com
% Organization:  
% 
% Copyright (c) 2010, 2011, 2012, Raniere Gaia Costa da Silva. All rights 
% reserved.
% 
% This file is license under the terms of a Creative Commons Attribution 
% 3.0 Unported License, or (at your option) any later version. More details
% at <http://creativecommons.org/licenses/by/3.0/>.
\begin{tabular}{cc|cc|cc}
    \hline
    Com. & Res. & Com. & Res. & Com. & Res. \\ \hline
    \lstinline!\Re! & $\Re$ & \lstinline!\Im! & $\Im$ & \lstinline!\nabla! & $\nabla$ \\
    \lstinline!\partial! & $\partial$ & \lstinline!\infty! & $\infty$ & \lstinline!\emptyset! & $\emptyset$ \\
    \lstinline!\varnothing! & $\varnothing$ & \lstinline!\forall! & $\forall$ & \lstinline!\exists! & $\exists$ \\
    \lstinline!\nexists! & $\nexists$ & \lstinline!\angle! & $\angle$ & \lstinline!\measuredangle! & $\measuredangle$ \\
    \lstinline!\sphericalangle! & $\sphericalangle$ & \lstinline!\top! & $\top$ & \lstinline!\bot! & $\bot$ \\
    \lstinline!\diagup! & $\diagup$ & \lstinline!\diagdown! & $\diagdown$ & \lstinline!\triangle! & $\triangle$ \\
    \lstinline!\triangledown! & $\triangledown$ & \lstinline!\blacktriangle! & $\blacktriangle$ & \lstinline!\blacktriangledown! & $\blacktriangledown$ \\
    \lstinline!\Diamond! & $\Diamond$ & \lstinline!\lozenge! & $\lozenge$ & \lstinline!\blacklozenge! & $\blacklozenge$ \\
    \lstinline!\bigstar! & $\bigstar$ & \lstinline!\Box! & $\Box$ & \lstinline!\square! & $\square$ \\
    \lstinline!\blacksquare! & $\blacksquare$ & \lstinline!\clubsuit! & $\clubsuit$ & \lstinline!\diamondsuit! & $\diamondsuit$ \\
    \lstinline!\heartsuit! & $\heartsuit$ & \lstinline!\spadesuit! & $\spadesuit$ \\ \hline
\end{tabular}

\end{table}

\begin{table}[h!tb]
    \centering
    \caption{Alfabeto Hebreu}
    \label{tab:math_hebrew}
    % Filename: math_hebrew@latex_with_vim.tex
% This code is part of LaTeX with Vim.
% 
% Description: LaTeX with Vim is free book about Vim, LaTeX and Git.
% 
% Created: 30.03.12 12:18:04 AM
% Last Change: 30.03.12 12:18:16 AM
% 
% Author: Raniere Gaia Costa da Silva, r.gaia.cs@gmail.com
% Organization:  
% 
% Copyright (c) 2010, 2011, 2012, Raniere Gaia Costa da Silva. All rights 
% reserved.
% 
% This file is license under the terms of a Creative Commons Attribution 
% 3.0 Unported License, or (at your option) any later version. More details
% at <http://creativecommons.org/licenses/by/3.0/>.
\begin{tabular}{cc|cc|cc|cc}
    \hline
    Com. & Res. & Com. & Res. & Com. & Res. & Com. & Res. \\ \hline
    \lstinline!\aleph! & $\aleph$ & \lstinline!\beth! & $\beth$ & \lstinline!\daleth! & $\daleth$ & \lstinline!\gimel! & $\gimel$ \\ \hline
\end{tabular}

\end{table}

\begin{table}[h!tb]
    \centering
    \caption{Alfabeto Grego, letras minúsculas}
    \label{tab:math_greek}
    % Filename: math_greek@latex_with_vim.tex
% This code is part of LaTeX with Vim.
% 
% Description: LaTeX with Vim is free book about Vim, LaTeX and Git.
% 
% Created: 30.03.12 12:17:35 AM
% Last Change: 30.03.12 12:17:39 AM
% 
% Author: Raniere Gaia Costa da Silva, r.gaia.cs@gmail.com
% Organization:  
% 
% Copyright (c) 2010, 2011, 2012, Raniere Gaia Costa da Silva. All rights 
% reserved.
% 
% This file is license under the terms of a Creative Commons Attribution 
% 3.0 Unported License, or (at your option) any later version. More details
% at <http://creativecommons.org/licenses/by/3.0/>.
\begin{tabular}{cc|cc|cc|cc}
    \hline
    Com. & Res. & Com. & Res. & Com. & Res. & Com. & Res. \\ \hline
    \lstinline!\alpha! & $\alpha$ & \lstinline!\beta! & $\beta$ & \lstinline!\gamma! & $\gamma$ & \lstinline!\delta! & $\delta$ \\
    \lstinline!\epsilon! & $\epsilon$ & \lstinline!\zeta! & $\zeta$ & \lstinline!\eta! & $\eta$ & \lstinline!\theta! & $\theta$ \\
    \lstinline!\iota! & $\iota$ & \lstinline!\kappa! & $\kappa$ & \lstinline!\lambda! & $\lambda$ & \lstinline!\mu! & $\mu$ \\
    \lstinline!\nu! & $\nu$ & \lstinline!\xi! & $\xi$ & \lstinline!\pi! & $\pi$ & \lstinline!\rho! & $\rho$ \\
    \lstinline!\sigma! & $\sigma$ & \lstinline!\tau! & $\tau$ & \lstinline!\upsilon! & $\upsilon$ & \lstinline!\phi! & $\phi$ \\
    \lstinline!\chi! & $\chi$ & \lstinline!\psi! & $\psi$ & \lstinline!\omega! & $\omega$ & \lstinline!\digamma! & $\digamma$ \\
    \lstinline!\varepsilon! & $\varepsilon$ & \lstinline!\vartheta! & $\vartheta$ & \lstinline!\varkappa! & $\varkappa$ & \lstinline!\varpi! & $\varpi$ \\
    \lstinline!\varrho! & $\varrho$ & \lstinline!\varsigma! & $\varsigma$ & \lstinline!\varphi! & $\varphi$ & & \\ \hline
\end{tabular}

\end{table}

\begin{table}[h!tb]
    \centering
    \caption{Alfabeto Grego, letras maiúsculo}
    \label{tab:math_greek_capital}
    % Filename: math_greek_capital@latex_with_vim.tex
% This code is part of LaTeX with Vim.
% 
% Description: LaTeX with Vim is free book about Vim, LaTeX and Git.
% 
% Created: 30.03.12 12:17:50 AM
% Last Change: 30.03.12 12:17:56 AM
% 
% Author: Raniere Gaia Costa da Silva, r.gaia.cs@gmail.com
% Organization:  
% 
% Copyright (c) 2010, 2011, 2012, Raniere Gaia Costa da Silva. All rights 
% reserved.
% 
% This file is license under the terms of a Creative Commons Attribution 
% 3.0 Unported License, or (at your option) any later version. More details
% at <http://creativecommons.org/licenses/by/3.0/>.
\begin{tabular}{cc|cc|cc|cc}
    \hline
    Com. & Res. & Com. & Res. & Com. & Res. & Com. & Res. \\ \hline
    \lstinline!\Gamma! & $\Gamma$ & \lstinline!\Delta! & $\Delta$ & \lstinline!\Theta! & $\Theta$ & \lstinline!\Lambda! & $\Lambda$ \\
    \lstinline!\Xi! & $\Xi$ & \lstinline!\Pi! & $\Pi$ & \lstinline!\Sigma! & $\Sigma$ & \lstinline!\Upsilon! & $\Upsilon$ \\
    \lstinline!\Phi! & $\Phi$ & \lstinline!\Psi! & $\Psi$ & \lstinline!\Omega! & $\Omega$ \\
    \lstinline!\varGamma! & $\varGamma$ & \lstinline!\varDelta! & $\varDelta$ & \lstinline!\varTheta! & $\varTheta$ & \lstinline!\varLambda! & $\varLambda$ \\
    \lstinline!\varXi! & $\varXi$ & \lstinline!\varPi! & $\varPi$ & \lstinline!\varSigma! & $\varSigma$ & \lstinline!\varUpsilon! & $\varUpsilon$ \\
    \lstinline!\varPhi! & $\varPhi$ & \lstinline!\varPsi! & $\varPsi$ &
    \lstinline!\varOmega! & $\varOmega$ & & \\ \hline
\end{tabular}

\end{table}
