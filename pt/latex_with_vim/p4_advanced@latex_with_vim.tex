% Filename: p4_advanced@latex_with_vim.tex
% This code is part of LaTeX with Vim.
% 
% Description: LaTeX with Vim is free book about Vim, LaTeX and Git.
% 
% Created: 29.03.12 11:50:16 PM
% Last Change: 29.03.12 11:51:14 PM
% 
% Author: Raniere Gaia Costa da Silva, r.gaia.cs@gmail.com
% Organization:  
% 
% Copyright (c) 2010, 2011, 2012, Raniere Gaia Costa da Silva. All rights 
% reserved.
% 
% This file is license under the terms of a Creative Commons Attribution 
% 3.0 Unported License, or (at your option) any later version. More details
% at <http://creativecommons.org/licenses/by/3.0/>.
\chapter{LaTeX-suite: Parte 2} \label{V:LaTeX2}
Neste capítulo apresentaremos alguns atalhos disponíveis no LaTeX-Suite.

\section{Fonte}
Três letras maiúsculas, onde a primeira é \textsf{F} e as duas outras uma identificação da fonte, são expandidads pelo LaTeX-Suite para o comando de troca do fonte do LaTeX. Na Tabela \ref{tab:vimlatex_font_shortcuts} apresenta-se alguns exemplos. 

\begin{table}[h!tb]
    \centering
    \caption{Atalhos disponíveis para fontes.}
    \label{tab:vimlatex_font_shortcuts}
    % Filename: vimlatex_font_shortcuts@latex_with_vim.tex
% This code is part of LaTeX with Vim.
% 
% Description: LaTeX with Vim is free book about Vim, LaTeX and Git.
% 
% Created: 30.03.12 12:19:38 AM
% Last Change: 30.03.12 12:19:44 AM
% 
% Author: Raniere Gaia Costa da Silva, r.gaia.cs@gmail.com
% Organization:  
% 
% Copyright (c) 2010, 2011, 2012, Raniere Gaia Costa da Silva. All rights 
% reserved.
% 
% This file is license under the terms of a Creative Commons Attribution 
% 3.0 Unported License, or (at your option) any later version. More details
% at <http://creativecommons.org/licenses/by/3.0/>.
\begin{tabular}{llll}
    \hline
    Atalho & Comando & Atalho & Comando \\ \hline
    \textsf{FIT} & \textbackslash\textsf{textit} & \textsf{FBF} & \textbackslash\textsf{textbf} \\
    \textsf{FRM} & \textbackslash\textsf{textrm} & \textsf{FSF} & \textbackslash\textsf{textsf} \\
    \textsf{FTT} & \textbackslash\textsf{texttt} & \textsf{FSC} & \textbackslash\textsf{textsc} \\ \hline
\end{tabular}

\end{table}

\section{Ambientes}
Algum ambientes também podem ser acessados pela combinação de três letras. Na Tabela \ref{tab:vimlatex_environment_shortcuts} apresentamos alguns deles.
\begin{table}[h!tb]
    \centering
    \caption{Atalhos disponíveis para ambientes.}
    \label{tab:vimlatex_environment_shortcuts}
    % Filename: vimlatex_environment_shortcuts@latex_with_vim.tex
% This code is part of LaTeX with Vim.
% 
% Description: LaTeX with Vim is free book about Vim, LaTeX and Git.
% 
% Created: 30.03.12 12:19:25 AM
% Last Change: 30.03.12 12:19:31 AM
% 
% Author: Raniere Gaia Costa da Silva, r.gaia.cs@gmail.com
% Organization:  
% 
% Copyright (c) 2010, 2011, 2012, Raniere Gaia Costa da Silva. All rights 
% reserved.
% 
% This file is license under the terms of a Creative Commons Attribution 
% 3.0 Unported License, or (at your option) any later version. More details
% at <http://creativecommons.org/licenses/by/3.0/>.
\begin{tabular}{llll}
    \hline
    Atalho & Ambiente & Atalho & Ambiente \\ \hline
    \textsf{EFL} & \textsf{flushleft} & \textsf{EFL} & \textsf{flushright} \\
    \textsf{EEA} & \textsf{eqnarray} & \textsf{EEQ} & \textsf{equation} \\
    \textsf{EQE} & \textsf{quote} & \textsf{EQN} & \textsf{quotation} \\ \hline
\end{tabular}

\end{table}

\section{Divisão do texto}
As divisões para o texto permitidas pelo LaTeX e apresentadas no Capítulo \ref{sch:latex:publi} também podem ser acessadas pela combinação de três letras como descrito na Tabela \ref{tab:vimlatex_sectioning_shortcuts}.
\begin{table}[h!tb]
    \centering
    \caption{Atalhos disponíveis para divisão.}
    \label{tab:vimlatex_sectioning_shortcuts}
    % Filename: vimlatex_sectioning_shortcuts@latex_with_vim.tex
% This code is part of LaTeX with Vim.
% 
% Description: LaTeX with Vim is free book about Vim, LaTeX and Git.
% 
% Created: 30.03.12 12:21:24 AM
% Last Change: 30.03.12 12:21:28 AM
% 
% Author: Raniere Gaia Costa da Silva, r.gaia.cs@gmail.com
% Organization:  
% 
% Copyright (c) 2010, 2011, 2012, Raniere Gaia Costa da Silva. All rights 
% reserved.
% 
% This file is license under the terms of a Creative Commons Attribution 
% 3.0 Unported License, or (at your option) any later version. More details
% at <http://creativecommons.org/licenses/by/3.0/>.
\begin{tabular}{llll}
    \hline
    Atalho & Divisão & Atalho & Divisão \\ \hline
    \textsf{SPA} & \textsf{part} & \textsf{SCH} & \textsf{chapter} \\
    \textsf{SSE} & \textsf{section} & \textsf{SSS} & \textsf{subsection} \\
    \textsf{SS2} & \textsf{subsubsection} & \textsf{SPG} & \textsf{paragraph} \\
    \textsf{SSP} & \textsf{subparagraph} & & \\ \hline
\end{tabular}

\end{table}

\section{Símbolos matemáticos}
O LaTeX-Suite apresenta alguns atalhos para alguns símbolos matemáticos e estes são apresentados na Tabela \ref{tab:vimlatex_math_shortcuts}.
\begin{table}[h!tb]
    \centering
    \caption{Atalhos para símbolos matemáticos.}
    \label{tab:vimlatex_math_shortcuts}
    % Filename: vimlatex_math_shortcuts@latex_with_vim.tex
% This code is part of LaTeX with Vim.
% 
% Description: LaTeX with Vim is free book about Vim, LaTeX and Git.
% 
% Created: 30.03.12 12:20:38 AM
% Last Change: 30.03.12 12:21:15 AM
% 
% Author: Raniere Gaia Costa da Silva, r.gaia.cs@gmail.com
% Organization:  
% 
% Copyright (c) 2010, 2011, 2012, Raniere Gaia Costa da Silva. All rights 
% reserved.
% 
% This file is license under the terms of a Creative Commons Attribution 
% 3.0 Unported License, or (at your option) any later version. More details
% at <http://creativecommons.org/licenses/by/3.0/>.
\begin{tabular}{ccc|ccc|ccc}
    \hline
    Atalho & Comando & Resultado & Atalho & Comando & Resultado & Atalho & Comando & Resultado \\ \hline
    `\textasciicircum & \textbackslash\textsf{hat} & & `\_ & \textbackslash\textsf{bar} & & `\textasciitilde & \textbackslash\textsf{tilde}  \\
    `; & \textbackslash\textsf{dot} & & `: & \textbackslash\textsf{ddot} & & `= & \textbackslash\textsf{equiv} & $\equiv$ \\
    `$8$ & \textbackslash\textsf{infty} & $\infty$ & `$6$ & \textbackslash\textsf{partial} & $\partial$ & `I & \textbackslash\textsf{int} \\
    `2 & \textbackslash\textsf{sqrt} & &  `/ ou `\% & \textbackslash\textsf{frac} & & `\textasteriskcentered & \textbackslash\textsf{times} & $\times$ \\
    `. & \textbackslash\textsf{cdot} & $\cdot$ & `@ & \textbackslash\textsf{circ} & $\circ$ & `0 & \textasciicircum\textbackslash\textsf{circ} & $^\circ$ \\
    `\textbackslash & \textbackslash\textsf{setminus} & $\setminus$ & `\& & \textbackslash\textsf{wedge} &  $\wedge$ & `- & \textbackslash\textsf{bigcap} &  $\bigcap$ \\
    `+ & \textbackslash\textsf{bigcup} & $\bigcup$ &  `( & \textbackslash\textsf{subset} & $\subset$ & `) & \textbackslash\textsf{supset} & $\supset$ \\
    `\textless & \textbackslash\textsf{le} & $\le$ & `\textgreater & \textbackslash\textsf{ge} & $\ge$ \\
    \hline
 \end{tabular}

\end{table}

E o atalho `, corresponde ao comando \textbackslash\textsf{nonumber}.

%No modo visual, estão disponíveis os atalhos apresentados na Tabela \ref{tab:math_mv}.
%\begin{table}[h!tb]
%\centering
%\caption{Atalhos disponíveis no modo visual.}
%\label{tab:math_mv}
%\begin{tabular}{cc|cc}
%\hline
%Atalho & Comando & Atalho & Comando \\ \hline
%`( &  \textbackslash\textsf{left}(  \textbackslash\textsf{right}) & `[ &  \textbackslash\textsf{left}[  \textbackslash\textsf{right}] \\
%`{ &  \textbackslash\textsf{left}\textbackslash\{  \textbackslash\textsf{right}\textbackslash\{ & `\$ &  \$\$ \$\$ \\
% \hline
%\end{tabular}
%\end{table}

\section{Alfabeto Grego}

\begin{table}[h!tb]
    \centering
    \caption{Atalhos para o Alfabeto Grego, letras minúsculas}
    \label{tab:vimlatex_greek_shortcuts}
    % Filename: vimlatex_greek_shortcuts@latex_with_vim.tex
% This code is part of LaTeX with Vim.
% 
% Description: LaTeX with Vim is free book about Vim, LaTeX and Git.
% 
% Created: 30.03.12 12:20:18 AM
% Last Change: 30.03.12 12:20:24 AM
% 
% Author: Raniere Gaia Costa da Silva, r.gaia.cs@gmail.com
% Organization:  
% 
% Copyright (c) 2010, 2011, 2012, Raniere Gaia Costa da Silva. All rights 
% reserved.
% 
% This file is license under the terms of a Creative Commons Attribution 
% 3.0 Unported License, or (at your option) any later version. More details
% at <http://creativecommons.org/licenses/by/3.0/>.
\begin{tabular}{ccc|ccc|ccc}
    \hline
    Atalho & Comando & Resultado & Atalho & Comando & Resultado & Atalho & Comando & Resultado \\ \hline
    `\textsf{a} & \textbackslash\textsf{alpha} & $\alpha$ & `\textsf{b} & \textbackslash\textsf{beta} & $\beta$ & `\textsf{g} & \textbackslash\textsf{gamma} & $\gamma$ \\
    `\textsf{d} & \textbackslash\textsf{delta} & $\delta$ & \textbackslash\textsf{epsilon} & $\epsilon$ & `\textsf{z} & & \textbackslash\textsf{zeta} & $\zeta$ \\
    `\textsf{h} & \textbackslash\textsf{eta} & $\eta$ & `\textsf{q} & \textbackslash\textsf{theta} & $\theta$ & & \textbackslash\textsf{iota} & $\iota$ \\
    `\textsf{k} & \textbackslash\textsf{kappa} & $\kappa$ & `\textsf{l} & \textbackslash\textsf{lambda} & $\lambda$ & `\textsf{m} & \textbackslash\textsf{mu} & $\mu$ \\
    `\textsf{n} & \textbackslash\textsf{nu} & $\nu$ & `\textsf{x} & \textbackslash\textsf{xi} & $\xi$ & `\textsf{p} & \textbackslash\textsf{pi} & $\pi$ \\
    `\textsf{r} & \textbackslash\textsf{rho} & $\rho$ & `\textsf{s} & \textbackslash\textsf{sigma} & $\sigma$ & `\textsf{t} & \textbackslash\textsf{tau} & $\tau$ \\
    `\textsf{u} & \textbackslash\textsf{upsilon} & $\upsilon$ & & \textbackslash\textsf{phi} & $\phi$ & `\textsf{c} & \textbackslash\textsf{chi} & $\chi$ \\
    `\textsf{y} & \textbackslash\textsf{psi} & $\psi$ & & \textbackslash\textsf{omega} & $\omega$ & & \textbackslash\textsf{digamma} & $\digamma$ \\
    `\textsf{e} & \textbackslash\textsf{varepsilon} & $\varepsilon$ & & \textbackslash\textsf{vartheta} & $\vartheta$ & & \textbackslash\textsf{varkappa} & $\varkappa$ \\
    & \textbackslash\textsf{varpi} & $\varpi$ & & \textbackslash\textsf{varrho} & $\varrho$ & `\textsf{v} & \textbackslash\textsf{varsigma} & $\varsigma$ \\
    `\textsf{f} & \textbackslash\textsf{varphi} & $\varphi$ & & \\ \hline
\end{tabular}

\end{table}

\begin{table}[h!tb]
    \centering
    \caption{Atalhos para o Alfabeto Grego, letras maiúsculo}
    \label{tab:vimlatex_greek_capital_shortcuts}
    % Filename: vimlatex_greek_capital_shortcuts@latex_with_vim.tex
% This code is part of LaTeX with Vim.
% 
% Description: LaTeX with Vim is free book about Vim, LaTeX and Git.
% 
% Created: 30.03.12 12:19:55 AM
% Last Change: 30.03.12 12:20:01 AM
% 
% Author: Raniere Gaia Costa da Silva, r.gaia.cs@gmail.com
% Organization:  
% 
% Copyright (c) 2010, 2011, 2012, Raniere Gaia Costa da Silva. All rights 
% reserved.
% 
% This file is license under the terms of a Creative Commons Attribution 
% 3.0 Unported License, or (at your option) any later version. More details
% at <http://creativecommons.org/licenses/by/3.0/>.
\begin{tabular}{ccc|ccc|ccc}
    \hline
    Atalho & Comando & Resultado & Atalho & Comando & Resultado & Atalho & Comando & Resultado \\ \hline
    `\textsf{G} & \textbackslash\textsf{Gamma} & $\Gamma$ & `\textsf{D} & \textbackslash\textsf{Delta} & $\Delta$ & `\textsf{T} & \textbackslash\textsf{Theta} & $\Theta$ \\
    `\textsf{L} & \textbackslash\textsf{Lambda} & $\Lambda$ & `\textsf{X} & \textbackslash\textsf{Xi} & $\Xi$ & & \textbackslash\textsf{Pi} & $\Pi$ \\
    `\textsf{S} & \textbackslash\textsf{Sigma} & $\Sigma$ & `\textsf{U} & \textbackslash\textsf{Upsilon} & $\Upsilon$ & `\textsf{F} & \textbackslash\textsf{Phi} & $\Phi$ \\
    `\textsf{Y} & \textbackslash\textsf{Psi} & $\Psi$ & `\textsf{W} & \textbackslash\textsf{Omega} & $\Omega$ \\
    & \textbackslash\textsf{varGamma} & $\varGamma$ & & \textbackslash\textsf{varDelta} & $\varDelta$ & & \textbackslash\textsf{varTheta} & $\varTheta$ \\
    & \textbackslash\textsf{varLambda} & $\varLambda$ & & \textbackslash\textsf{varXi} & $\varXi$ & & \textbackslash\textsf{varPi} & $\varPi$ \\
    & \textbackslash\textsf{varSigma} & $\varSigma$ & & \textbackslash\textsf{varUpsilon} & $\varUpsilon$ & & \textbackslash\textsf{varPhi} & $\varPhi$ \\
    & \textbackslash\textsf{varPsi} & $\varPsi$ & & \textbackslash\textsf{varOmega} & $\varOmega$ & & \\ \hline
\end{tabular}

\end{table}

\section{\textsf{Ctrl} \textit{key}}
Para utilizar um limitador juntamente com os comandos \textbackslash\textsf{left} e \textbackslash\textsf{right} deve-se inserir o delimitador e em seguida pressionar \textsf{Ctrl}+\textsf{l}. São permitido os seguinte limitadores: (, [, \{, \textbar e \textless.

Caso seja pressionado \textsf{Ctrl}+\textsf{l} na ausência de um limitador é inserido o comando \textbackslash\textsf{label}.

Já quando estiver dentro do ambiente \textsf{itemize} ou \textsf{enumerate} ao pressionar \textsf{Ctrl}+\textsf{l} o comando \textbackslash\textsf{item} é inserido.

\section{\textit{Smart key}}
Ao pressionar "(aspas dublas) o LaTeX-Suite troca-as por `` seguido por '' e ...(três pontos) por \textbackslash\textsf{ldots}, quando fora do modo matemático, ou \textbackslash\textsf{cdotes}, quando no modo matemático.
