% Filename: p3_algorithm@latex_with_vim.tex
% This code is part of LaTeX with Vim.
% 
% Description: LaTeX with Vim is free book about Vim, LaTeX and Git.
% 
% Created: 29.03.12 11:33:38 PM
% Last Change: 30.03.12 12:08:42 AM
% 
% Author: Raniere Gaia Costa da Silva, r.gaia.cs@gmail.com
% Organization:  
% 
% Copyright (c) 2010, 2011, 2012, Raniere Gaia Costa da Silva. All rights 
% reserved.
% 
% This file is license under the terms of a Creative Commons Attribution 
% 3.0 Unported License, or (at your option) any later version. More details
% at <http://creativecommons.org/licenses/by/3.0/>.
\chapter{Códigos e algoritmos}
Quando precisa-se incluir um algoritmo ou código o LaTeX fornece várias opções.

\section{\textsf{algorithmic}}

O pacote \textsf{algorithmic} contem alguns ambientes próprios para escrita de algoritmos.

\subsection{Inserindo um algoritmo}

O ambiente \textsf{algorithmic} é apropriado para escrever um algoritmo e os principais comandos são apresentados na Tabela \ref{tab:algorithmic}.
\begin{table}[h!tb]
    \centering
    \caption{Comandos disponíveis no ambiente \textsf{algorithmic}.}
    \label{tab:algorithmic}
    % Filename: algorithmic@latex_with_vim.tex
% This code is part of LaTeX with Vim.
% 
% Description: LaTeX with Vim is free book about Vim, LaTeX and Git.
% 
% Created: 30.03.12 12:11:31 AM
% Last Change: 30.03.12 12:11:38 AM
% 
% Author: Raniere Gaia Costa da Silva, r.gaia.cs@gmail.com
% Organization:  
% 
% Copyright (c) 2010, 2011, 2012, Raniere Gaia Costa da Silva. All rights 
% reserved.
% 
% This file is license under the terms of a Creative Commons Attribution 
% 3.0 Unported License, or (at your option) any later version. More details
% at <http://creativecommons.org/licenses/by/3.0/>.
\begin{tabular}{p{0.57\linewidth}p{0.4\linewidth}}
    \hline
    Código & Descrição \\ \hline
    \textbackslash\textsf{STATE} \textless\textsf{texto}\textgreater & Execução de um comando. \\
    \textbackslash\textsf{IF}\{\textless\textsf{cond}\textgreater\} \textless\textsf{texto}\textgreater \ \textbackslash\textsf{ENDIF} & Condicional \textit{if}. \\
    \textbackslash\textsf{IF}\{\textless\textsf{cond}\textgreater\} \textless\textsf{texto}\textgreater \ \textbackslash\textsf{ELSE} \textless\textsf{texto}\textgreater \ \textbackslash\textsf{ENDIF} & Condicional \textit{if-else}. \\
    \textbackslash\textsf{IF}\{\textless\textsf{cond}\textgreater\} \textless\textsf{texto}\textgreater \ \textbackslash\textsf{ELSEIF}\{\textless\textsf{cond}\textgreater\} \textless\textsf{texto}\textgreater \ \textbackslash\textsf{ELSE} \textless\textsf{texto}\textgreater \ \textbackslash\textsf{ENDIF} & Condicional \textit{if-else}, o comando \textbackslash\textsf{ELSEIF} pode ser utilizado mais de uma vez. \\
    \textbackslash\textsf{FOR}\{\textless\textsf{cond}\textgreater\} \textless\textsf{texto}\textgreater \ \textbackslash\textsf{ENDFOR} & Loop \textit{for}. \\
    \textbackslash\textsf{FORALL}\{\textless\textsf{cond}\textgreater\} \textless\textsf{texto}\textgreater \ \textbackslash\textsf{ENDFOR} & Loop \textit{for all}. \\
    \textbackslash\textsf{WHILE}\{\textless\textsf{cond}\textgreater\} \textless\textsf{texto}\textgreater \ \textbackslash\textsf{ENDWHILE} & Loop \textit{while}. \\
    \textbackslash\textsf{REPEAT} \textless\textsf{texto}\textgreater \ \textbackslash\textsf{UNTIL}\{\textless\textsf{cond}\textgreater\}  & Loop \textit{repeat-until}. \\
    \textbackslash\textsf{LOOP} \textless\textsf{texto}\textgreater \ \textbackslash\textsf{ENDLOOP} & Loop infinito. \\
    \textbackslash\textsf{TO} & Equivalente a ``até'', para indicar intervalos. \\
    \textbackslash\textsf{TRUE} & Variável booleana. \\
    \textbackslash\textsf{FALSE} & Variável booleana. \\
    \textbackslash\textsf{AND} & Operador lógico \textit{and}. \\
    \textbackslash\textsf{OR} & Operador lógico \textit{or}. \\
    \textbackslash\textsf{XOR} & Operador lógico \textit{xor}. \\
    \textbackslash\textsf{NOT} & Operador lógico \textit{not}. \\
    \textbackslash\textsf{REQUIRE} \textless\textsf{texto}\textgreater & Condicional. \\
    \textbackslash\textsf{ENSURE} \textless\textsf{texto}\textgreater & Teste de resultado. \\
    \textbackslash\textsf{RETURN} \textless\textsf{texto}\textgreater & Valor de retorno. \\
    \textbackslash\textsf{PRINT} \textless\textsf{texto}\textgreater & Imprime um texto. \\
    \textbackslash\textsf{COMMENT} \textless\textsf{texto}\textgreater & Comentário. \\ \hline
\end{tabular}

\end{table}

O código abaixo
\begin{latexcode}
    \begin{algorithmic}
        \STATE $a \leftarrow 1$
        \IF{$a$ is even}
        \PRINT ``$a$ is even''
        \ELSE
        \PRINT ``$a$ is odd''
        \ENDIF
    \end{algorithmic}
\end{latexcode}
produz o seguinte exemplo
\begin{algorithmic}
    \STATE $a \leftarrow 1$
    \IF{$a$ is even}
    \PRINT ``$a$ is even''
    \ELSE
    \PRINT ``$a$ is odd''
    \ENDIF
\end{algorithmic}

Como podemos observar pelo exemplo a identação do algoritmo é feita de maneira automática.

Para algoritmos muito extensos é recomendado numerar as linhas o que pode ser feito utilizando-se do comando
\begin{latexcode}
    \begin{algorithmic}[linha]
\end{latexcode}
onde \textsf{linha} é um inteiro de forma que as linhas que são múltiplas de \textsf{linha} serão numeradas.

\subsection{\textsf{algorithm}}

O pacote \textsf{algorithm} permite trabalhar algoritmos como um objeto flutuante quando utilizado o ambiente \textsf{algorithm}.

O código abaixo
\begin{latexcode}
    \begin{algorithm}
        \caption{Exemplo do ambiente \textsf{algorithm}.}
        \label{alg:exem}
        \begin{algorithmic}
            \STATE $a \leftarrow 1$
            \IF{$a$ is even}
            \PRINT ``$a$ is even''
            \ELSE
            \PRINT ``$a$ is odd''
            \ENDIF
        \end{algorithmic}
    \end{algorithm}
\end{latexcode}
produz o seguinte exemplo
\begin{algorithm}
    \caption{Exemplo do ambiente \textsf{algorithm}.}
    \label{alg:exem}
    \begin{algorithmic}
        \STATE $a \leftarrow 1$
        \IF{$a$ is even}
        \PRINT ``$a$ is even''
        \ELSE
        \PRINT ``$a$ is odd''
        \ENDIF
    \end{algorithmic}
\end{algorithm}

\section{Códigos}

Para inserir código recomenda-se o pacote \textsf{listings}\footnote{Este pacote não funciona adequadamente com a codificação UFT8 de modo que é recomendado utilizar a codificação Latin1. Aos que utilizarem a codificação UFT8 podem utilizar o comando \textsf{literate}(para maiores detalhes olhar \cite{Moses:2007:Listings}).} que contem o ambiente \textsf{lstlisting}, este funciona de maneira semelhante ao ambiente \textsf{verbatim} mas com suporte a \textit{highlighting}.

A seguir apresentamos um exemplo do ambiente \textsf{lstlisting}.
\begin{minipage}[t]{0.47\linewidth}
    \textbackslash begin\{lstlisting\} \\
    void main()\{ \\
    print("Hello world!"); \\
    \} \\
    \textbackslash end\{lstlisting\}
\end{minipage} \hfill
\begin{minipage}[t]{0.47\linewidth}
    \begin{lstlisting}
    void main(){
    print("Hello world!");
    }
    \end{lstlisting}
\end{minipage}

No pacote \textsf{listings} também é definido o comando \textbackslash\textsf{lstinputlisting} que tem como parâmetro o nome do arquivo com o código a ser inserido.

Por último, o pacote \textsf{listings} defini algumas opções, apresentadas na Tabela \ref{tab:listings}, que podem ser utilizadas tanto no ambiente \textsf{lstlisting}, como no comando \textbackslash\textsf{lstinputlisting} ou definidas no comando \textbackslash\textsf{lstset}.

\begin{table}[h!tb]
    \centering
    \caption{Opções disponibilizados pelo pacote \textsf{listings}.}
    \label{tab:listings}
\end{table}
