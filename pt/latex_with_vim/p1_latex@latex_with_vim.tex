% Filename: p1_latex@latex_with_vim.tex
% This code is part of LaTeX with Vim.
% 
% Description: LaTeX with Vim is free book about Vim, LaTeX and Git.
% 
% Created: 29.03.12 11:29:58 PM
% Last Change: 29.03.12 11:30:10 PM
% 
% Author: Raniere Gaia Costa da Silva, r.gaia.cs@gmail.com
% Organization:  
% 
% Copyright (c) 2010, 2011, 2012, Raniere Gaia Costa da Silva. All rights 
% reserved.
% 
% This file is license under the terms of a Creative Commons Attribution 
% 3.0 Unported License, or (at your option) any later version. More details
% at <http://creativecommons.org/licenses/by/3.0/>.
\chapter{\LaTeX}
O texto a seguir foi adaptado da Wikipedia (\url{http://en.wikipedia.org/wiki/TeX}).

\TeX\ ou TeX é um sistema de tipografia desenvolvido e em sua maior parte escrita por Donald Knuth. Foi desenvolvido com dois objetivos principais em mente: permitir qualquer pessoa produzir livros de alta qualidade usando uma quantidade razoável de esforço, e providenciar um sistema que produza exatamente o mesmo resultado em todo computador, hoje e no futuro.

TeX é um meio popular para tipografia de fórmulas matemáticas complexas.

O texto a seguir foi adaptado da Wikipedia (\url{http://en.wikipedia.org/wiki/LaTeX}).

\LaTeX\ ou LaTeX é uma linguagem de marcação e preparativo do sistema para o TeX. O termo LaTeX refere-se apenas a linguagem na qual o documento é escrita. Para a criação de um documento em LaTeX, um arquivo com a extensão \textsf{.tex} deve ser criado utilizando algum editor de texto.

LaTeX tenta prover uma linguagem de alto nível, para acessar o poder do TeX. LaTeX essencialmente comprime uma coleção de macros originais do TeX e um programa para processar o documento LaTeX. Devido aos comandos TeX serem de baixo nível é usualmente mais simples para o usuário final utilizar o LaTeX.

LaTeX foi originalmente escrito por Leslie Lamport e tornou-se o método dominante para uso do TeX.

Atualmente, o Latex encontra-se na versão \LaTeXe\ ou LaTeX2e e é distribuido de acordo com os termos do LaTeX Project Public License (LPPL), sendo um software livre.
