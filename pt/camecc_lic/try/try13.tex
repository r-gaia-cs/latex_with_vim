% Filename: try13.tex
% This code is part of 'Cursos CAMECC: Introducao ao LaTeX para o Curso 29 - Licenciatura em Matematica'
% 
% Description: This file correspond exercise 13 of the textbook using in the course.
% 
% Created: 07.06.12 11:30:10 AM
% Last Change: 07.06.12 11:30:30 AM
% 
% Authors:
% - Raniere Silva, r.gaia.cs@gmail.com
% 
% Organization: CAMECC - Centro Academico dos Estudantes do IMECC
% 
% Copyright (c) 2012, Raniere Silva. All rights reserved.
% 
% This work is licensed under the Creative Commons Attribution-ShareAlike 3.0 Unported License. To view a copy of this license, visit http://creativecommons.org/licenses/by-sa/3.0/ or send a letter to Creative Commons, 444 Castro Street, Suite 900, Mountain View, California, 94041, USA.
%
% This work is distributed in the hope that it will be useful, but WITHOUT ANY WARRANTY; without even the implied warranty of MERCHANTABILITY or FITNESS FOR A PARTICULAR PURPOSE.
%
\documentclass[12pt, answers]{exam}
\usepackage[utf8]{inputenc}
\usepackage[a4paper]{geometry}
\usepackage{url}

\begin{document}
As quest\~{o}es apresentadas a seguir foram retiradadas do vestibular de 2012 da Universidade Estadual de Campinas - UNICAMP.
\begin{questions}
\question \begin{quote}
Há notícias que são de interesse público e há notícias que são de interesse do público. Se a celebridade ``x'' está saindo com o ator ``y'', isso não tem nenhum interesse público. Mas, dependendo de quem sejam ``x'' e ``y'', é de enorme interesse do público, ou de um certo público (numeroso), pelo menos.

As decisões do Banco Central para conter a inflação têm óbvio interesse público. Mas quase não despertam interesse, a não ser dos entendidos.

O jornalismo transita entre essas duas exigências, desafiado a atender às demandas de uma sociedade ao mesmo tempo massificada e segmentada, de um leitor que gravita cada vez mais apenas em torno de seus interesses particulares.

\begin{flushright}
(Fernando Barros e Silva, O jornalista e o assassino. Folha de São Paulo (versão on line), 18/04/2011. Acessado em 20/12/2011.)
\end{flushright}
\end{quote}
\begin{parts}
\part A palavra público é empregada no texto ora como substantivo, ora como adjetivo. Exemplifique cada um desses empregos com passagens do próprio texto e apresente o critério que você utilizou para fazer a distinção.
\begin{solution}
Na expressão ``interesse público'', público é adjetivo; em ``interesse do público'', público é substantivo. Como adjetivo, público concorda com o
substantivo a que se refere em gênero e número
(assim, se o substantivo fosse feminino e plural,
adjetivo assumiria as mesmas flexões: ``tendências
públicas''). Como substantivo, no segundo caso,
público forma uma locução adjetiva com a preposição de.
\end{solution}
\part Qual é, no texto, a diferença entre o que é chamado de interesse público e o que é chamado de interesse do público?
\begin{solution}
Interesse público é sinônimo de ``interesse social''
e refere algo que diz respeito à coletividade e a
afeta. Interesse do público designa o que desperta
curio sidade coletiva, ainda que tal curiosidade não
seja motivada por interesses justificáveis ou
legítimos.
\end{solution}
\end{parts}


\question O parágrafo reproduzido abaixo introduz a crônica intitulada Tragédia concretista, de Luís Martins.
 \begin{quote}
O poeta concretista acordou inspirado. Sonhara a noite toda com a namorada. E pensou: lábio, lábia. O lábio em que pensou era o da namorada, a lábia era a própria. Em todo o caso, na pior das hipóteses, já tinha um bom começo de poema. Todavia, cada vez mais obcecado pela lembrança daqueles lábios, achou que podia aproveitar a sua lábia e, provisoriamente desinteressado da poesia pura, resolveu telefonar à criatura amada, na esperança de maiores intimidades e vantagens. Até os poetas concretistas podem ser homens práticos.

\begin{flushright}
(Luís Martins, Tragédia concretista, em As cem melhores crônicas brasileiras. Rio de Janeiro: Objetiva, 2007, p. 132.)
\end{flushright}
\end{quote}
\begin{parts}
\part Compare lábio e lábia quanto à forma e ao significado. Considerando a especificidade do poeta, justifique a ocorrência dessas duas palavras dentro da crônica.
\begin{solution}
O jogo com lábia e lábio corresponde à figura
chamada paronomásia, que consiste na aproxi -
mação de palavras que apresentam semelhança
sonora e diferença de sentido. A paronomásia -
que é, segundo Roman Jakobson, figura central
na poesia em geral - foi eleita como princípio
construtivo de muitos poemas ditos concretos, a
ponto de ser identificada, por autores como Luís
Martins, com a essência da chamada poesia
concreta.
\end{solution}
\part Explique por que a palavra todavia (linha 3) é usada para introduzir um dos enunciados da crônica.
\begin{solution}
Todavia, conjunção adversativa, introduz o
período em que, contrariando a indicação anterior
sobre a confecção de um poema (``já tinha um bom
começo de poema''), o cronista relata a iniciativa
erótica que a paronomásia mencionada sugere ao
``poeta concretista''.
\end{solution}
\end{parts}

\question Glow sticks ou light sticks são pequenos tubos plásticos
utilizados em festas por causa da luz que eles emitem. Ao
serem pressionados, ocorre uma mistura de peróxido de
hidrogênio com um éster orgânico e um corante. Com o
tempo, o peróxido e o éster vão reagindo, liberando
energia que excita o corante, que está em excesso. O
corante excitado, ao voltar para a condição não excitada,
emite luz. Quanto maior a quantidade de moléculas
excitadas, mais intensa é a luz emitida. Esse processo é
contínuo, enquanto o dispositivo funciona. Com base no
conhecimento químico, é possível afirmar que o
funcionamento do dispositivo, numa temperatura mais
baixa, mostrará uma luz
\begin{choices}
\choice mais intensa e de menor duração que numa temperatura mais alta.
\choice mais intensa e de maior duração que numa
temperatura mais alta.
\CorrectChoice menos intensa e de maior duração que numa
temperatura mais alta.
\choice menos intensa e de maior duração que numa
temperatura mais alta.
\end{choices}

\question \begin{quote}
``Ninguém é mais do que eu partidário de uma política
exterior baseada na amizade íntima com os Estados
Unidos. A Doutrina Monroe impõe aos Estados Unidos
uma política externa que se começa a desenhar. (…) Em
tais condições a nossa diplomacia deve ser principalmente
feita em Washington (...). Para mim a Doutrina Monroe
(...) significa que politicamente nós nos desprendemos da
Europa tão completamente e definitivamente como a lua
da terra.''

\begin{flushright}
(Adaptado de Joaquim Nabuco, citado por José Maria de Oliveira Silva,
``Manoel Bonfim e a ideologia do imperialismo na América Latina'', em
Revista de História, n. 138. São Paulo, jul. 1988, p.88.
\end{flushright}
\end{quote}

Sobre o contexto ao qual o político e diplomata brasileiro
Joaquim Nabuco se refere, é possível afirmar que:
\begin{choices}
\CorrectChoice A Doutrina Monroe a que Nabuco se refere,
estabelecida em 1823, tinha por base a ideia de ``a
América para os americanos''.
\choice Joaquim Nabuco, em sua atuação como embaixador,
antecipou a política imperialista americana de tornar o
Brasil o ``quintal'' dos Estados Unidos.
\choice Ao declarar que a América estava tão distante da
Europa ``como a lua da terra'', Nabuco reforçava a
necessidade imediata de o Brasil romper suas relações
diplomáticas com Portugal.
\choice O pensamento americano considerava legítimas as
intenções norte-americanas na América Central, bem
como o apoio às ditaduras na América do Sul, desde o
século XIX.
\end{choices}
\end{questions}
\end{document}
