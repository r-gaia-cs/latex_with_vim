\begin{tabular}{llp{0.6\textwidth}}
    \hline
    Função & Código & Descrição \\ \hline
    \multirow{4}{*}{Tamanho} &  & Utiliza, por padrão, o tamanho 10. \\
    & \lcode{10pt} & Tamanho 10. \\
    & \lcode{11pt} & Tamanho 11. \\
    & \lcode{12pt} & Tamanho 12. \\ \hline
    \multirow{7}{*}{Papel} & & Utiliza, por padrão, o tamanho da folha correspondente carta. \\
    & \lcode{letterpaper} & Tamanho da folha correspondente carta. \\
    & \lcode{a4paper} & Tamanho da folha correspondente a A4. \\
    & \lcode{a5paper} & Tamanho da folha correspondente a A5. \\
    & \lcode{b5paper} & Tamanho da folha correspondente a B5. \\
    & \lcode{executivepaper} & Tamanho da folha correspondente a folha executiva. \\
    & \lcode{legalpaper} & Tamanho da folha correspondente a folha legal. \\ \hline
    \multirow{2}{*}{Al. equação} & & Por padrão centra as equações. \\
    & \lcode{fleqn} & Alinha as equações à esquerda. \\ \hline
    \multirow{2}{*}{Nº equação} & & Por padrão enumera as equações à direita. \\
    & \lcode{leqno} & Enumera as equações à esquerda. \\ \hline
    \multirow{4}{*}{Título} & & Por padrão a classe \lcode{article} não começa uma nova página após o título, enquanto que \lcode{report} e \lcode{book} o fazem. \\
    & \lcode{titlepage} & Começa uma nova página após o título. \\
    & \lcode{leqno} & Não começa uma nova página após o título. \\ \hline
    \multirow{4}{*}{Faces} & & Por padrão a classe \lcode{article} e \lcode{report} são a uma face e a classe \lcode{book} é a duas. \\
    & \lcode{oneside} & Gera o documento a uma face. \\
    & \lcode{twoside} & Gera o documento a duas fazes. \\ \hline
    \multirow{5}{*}{Começo} & & Não funciona com a classe \lcode{article} por nesta não existirem capítulos e por padrão a classe \lcode{report} começa os capítulos na próxima página disponível e a classe \lcode{book} sempre nas páginas à direita. \\
    & \lcode{openright} & Começa os capítulos sempre nas páginas à direita. \\
    & \lcode{openany} & Começa os capítulos na próxima página disponível. \\ \hline
    Colunas & \lcode{twocolumn} & Gera o arquivo utilizando-se de duas colunas. \\
    \hline
\end{tabular}
