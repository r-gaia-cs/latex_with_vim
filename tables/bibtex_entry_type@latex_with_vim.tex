% Filename: bibtex_entry_type@latex_with_vim.tex
% This code is part of LaTeX with Vim.
% 
% Description: LaTeX with Vim is free book about Vim, LaTeX and Git.
% 
% Created: 30.03.12 12:12:14 AM
% Last Change: 30.03.12 12:12:19 AM
% 
% Author: Raniere Gaia Costa da Silva, r.gaia.cs@gmail.com
% Organization:  
% 
% Copyright (c) 2010, 2011, 2012, Raniere Gaia Costa da Silva. All rights 
% reserved.
% 
% This file is license under the terms of a Creative Commons Attribution 
% 3.0 Unported License, or (at your option) any later version. More details
% at <http://creativecommons.org/licenses/by/3.0/>.
\begin{tabular}{lp{0.8\textwidth}}
    \hline
    Código & Descrição \\ \hline
    \textsf{article} & Um artigo presente em algum periódico, revista, jornal que forme uma unidade própria e possua título. \\
    \textsf{book} & Um livro com um ou mais autores que levam crédito pela obra. \\
    \textsf{inbook} & Uma parte de um livro que forme uma unidade própria e possua título. \\
    %\textsf{suppbook} & Um material suplementar de um livro. \\
    \textsf{booklet} & Material com as características de um livro, mas que não foi formalmente publicado. \\
    %\textsf{collection} & Um livro composto dos trabalhos de vários autores, normalmente possui um editor. \\
    \textsf{incollection} & Uma parte de um livro composto dos trabalhos de vários autores, normalmente possui um editor. \\
    %\textsf{suppcollection} & Um suplemento de um livro composto dos trabalhos de vários autores, normalmente possui um editor. \\
    \textsf{proceedings} & Uma palestra de uma conferência. \\
    \textsf{inproceedings} & Um artigo apresentado em uma conferência. \\
    %\textsf{periodical} & Um periódico em sua totalidade. \\
    %\textsf{suppperiodical} & Um suplemento de um periódico. \\
    \textsf{manual} & Um documento técnico, pode não estar disponível em versão impressa. \\
    \textsf{techreport} & Um documento técnico produzido por uma instituição de ensino, comércio \dots \\
    \textsf{mastersthesis} & Uma tese de mestrado escrita para uma instituição de ensino. \\
    \textsf{phdthesis} & Uma tese de doutorado escrita para uma instituição de ensino. \\
    %\textsf{thesis} & Uma tese escrita para uma instituição de ensino como requisito para uma formação. \\
    %\textsf{report} &  \\
    %\textsf{patent} & Uma patente ou requerimento de patente. \\
    %\textsf{reference} & Uma enciclopédia ou dicionário. \\
    \textsf{unpublished} & Um trabalho que não foi formalmente publicado, como um manuscrito. \\
    %\textsf{online} & Um documento disponível apenas on-line, como por exemplo um site. \\
    \textsf{misc} & Utilizado quando a obra não se encaixa nos \textsf{tipo}'s anteriores.
\end{tabular}
